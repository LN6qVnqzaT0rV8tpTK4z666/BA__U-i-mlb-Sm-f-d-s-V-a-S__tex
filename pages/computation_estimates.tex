\pagebreak

\section*{Aufwandsschätzung für Evidential Deep Learning}
\label{sec:aufwand-edl-amini}

Zur Reproduktion der Experimente aus \parencite{amini2020deep} ergibt sich folgender Rechenaufwand.

\begin{table}[!htbp]
  \centering
  \footnotesize
  \begin{tabularx}{\textwidth}{|l|X|X|}
    \hline
    \textbf{Kategorie} & \textbf{Inhalt} & \textbf{Yacht Beispielwert} \\
    \hline
    \textbf{Symbol} & Parameterdefinitionen & - \\
    \hline
    $D$ & Zahl der Datensätze (typ. 1-9) & - \\
    \hline
    $T$ & Zahl der Trainer-Varianten & 3 \\
    \hline
    $R$ & Zahl der Trials & 20 \\
    \hline
    $E$ & Zahl der Epochen & 40 \\
    \hline
    $N$ & Zahl der Trainingsdaten & 308 \\
    \hline
    $B$ & Batchgröße & 16 \\
    \hline
    $P$ & Modellparameteranzahl & 3 \\
    \hline
    \textbf{MLP Architektur} & Input Layer: $d_\text{in}=6$ \newline Hidden Layer: 2 x 50 Neuronen \newline Output Layer: 4 Neuronen & - \\
    \hline
    \textbf{Speicherbedarf} & Modellgewichte: $P \times 4$ Bytes \newline Aktivierungen: $\mathcal{O}(B \times L)$ \newline Datensätze: gering (UCI) & - \\
    \hline
    \textbf{Hardware} & Kleine Datensätze: CPU ausreichend \newline Große Datensätze: GPU empfohlen (z. B. RTX 3060, $\geq$ 6GB VRAM) & - \\
    \hline
  \end{tabularx}
  \caption{\footnotesize Übersicht Parameter, Architektur, Hardwareempfehlungen f. Reproduzierbarkeit Amini 2020}
  \label{tab:edl-gesamt}
\end{table}


\vspace{-0.5em}

\textbf{Anzahl Läufe:}
\[
\text{\#Total\_Runs} = D \times T \times R
\]

\textbf{Aufwand pro Lauf:}
\[
\mathcal{O}\biggl(E \cdot \frac{N}{B} \cdot P \cdot B \biggr) 
=
\mathcal{O}(E \cdot N \cdot P)
\]

\textbf{Gesamtaufwand:}
\[
\boxed{
\mathcal{O}\left( D \cdot T \cdot R \cdot E \cdot N \cdot P \right)
}
\]

\textbf{Berechnung FLOPs Yacht:}
\[
40 \times 20 \times 3{,}104 \times 16 
\approx 40 \times 20 \times 50,000 
= 40 \times 10^6 \text{ FLOPs}
\]

\[
\text{Gesamt: } 
60 \times 40 \times 10^6 
= 2.4 \times 10^9 \text{ FLOPs}
\]

\noindent
Die Experimente aus \parencite{amini2020deep} sind damit auch auf moderner Consumer-Hardware grundsätzlich reproduzierbar.
