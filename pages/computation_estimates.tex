\pagebreak

\section*{Aufwandsschätzung für Evidential Deep Learning}
\label{sec:aufwand-edl-amini}

Zur Reproduktion der Experimente aus \parencite{amini2020deep} ergibt sich folgender Rechenaufwand.

\vspace{0.5em}

\begin{table}[!htbp]
  \centering
  \begin{tabular}{ll}
    \toprule
    \textbf{Symbol} & \textbf{Bedeutung} \\
    \midrule
    $D$ & Zahl der Datensätze (1–9) \\
    $T$ & Zahl der Trainer-Varianten (3) \\
    $R$ & Zahl der Trials (z.\,B.\ 20) \\
    $E$ & Zahl der Epochen (z.\,B.\ 40) \\
    $N$ & Zahl der Trainingsdaten \\
    $B$ & Batchgröße \\
    $P$ & Modellparameteranzahl \\
    \bottomrule
  \end{tabular}
  \caption{Parameterdefinitionen}
\end{table}

\vspace{-0.5em}

\textbf{Anzahl Läufe:}
\[
\text{\#Total\_Runs} = D \times T \times R
\]

\textbf{Aufwand pro Lauf:}
\[
\mathcal{O}\biggl(E \cdot \frac{N}{B} \cdot P \cdot B \biggr) 
=
\mathcal{O}(E \cdot N \cdot P)
\]

\textbf{Gesamtaufwand:}
\[
\boxed{
\mathcal{O}\left( D \cdot T \cdot R \cdot E \cdot N \cdot P \right)
}
\]

\vspace{1em}

\noindent
\textbf{Beispiel Yacht-Dataset}

\vspace{0.3em}

\begin{table}[!htbp]
  \centering
  \begin{tabular}{lr}
    \toprule
    \textbf{Parameter} & \textbf{Wert} \\
    \midrule
    $N$ & 308 \\
    $B$ & 16 \\
    $E$ & 40 \\
    $T$ & 3 \\
    $R$ & 20 \\
    $P$ & 3,104 \\
    \bottomrule
  \end{tabular}
  \caption{Yacht-Dataset Parameter}
\end{table}

\vspace{-0.5em}

\textbf{Berechnung FLOPs Yacht:}
\[
40 \times 20 \times 3{,}104 \times 16 
\approx 40 \times 20 \times 50,000 
= 40 \times 10^6 \text{ FLOPs}
\]

\[
\text{Gesamt: } 
60 \times 40 \times 10^6 
= 2.4 \times 10^9 \text{ FLOPs}
\]

\vspace{0.5em}

\noindent
\textbf{MLP-Architektur Yacht}

\vspace{0.3em}

\begin{table}[!htbp]
  \centering
  \begin{tabular}{ll}
    \toprule
    \textbf{Layer} & \textbf{Konfiguration} \\
    \midrule
    Input & $d_\text{in} = 6$ \\
    Hidden & 2 × 50 Neuronen \\
    Output & 4 Neuronen \\
    \bottomrule
  \end{tabular}
  \caption{MLP-Struktur für Yacht}
\end{table}

\vspace{1em}

\noindent
\textbf{Speicherbedarf (ungefähr):}

\vspace{0.5em}

\begin{table}[!htbp]
  \centering
  \begin{tabular}{ll}
    \toprule
    \textbf{Komponente} & \textbf{Speicher} \\
    \midrule
    Modellgewichte & $P \times 4$ Bytes \\
    Aktivierungen & $\mathcal{O}(B \times L)$ \\
    Datensätze & gering bei UCI-Daten \\
    \bottomrule
  \end{tabular}
  \caption{Speicherbedarf beim Training}
\end{table}

\vspace{1em}

\noindent
\textbf{Hardware-Empfehlung:}

\vspace{0.5em}

\begin{table}[!htbp]
  \centering
  \begin{tabular}{p{0.4\textwidth}p{0.5\textwidth}}
    \toprule
    \textbf{Kleine Datensätze} & CPU ausreichend, 2–4 Kerne, 8 GB RAM \\
    \midrule
    \textbf{Große Datensätze} & GPU empfohlen (z.\,B.\ RTX 3060 oder höher, $\geq$ 6 GB VRAM) \\
    \bottomrule
  \end{tabular}
  \caption{Hardware-Empfehlungen}
\end{table}

\vspace{0.5em}

\noindent
Die Experimente aus \parencite{amini2020deep} sind damit auch auf moderner Consumer-Hardware grundsätzlich reproduzierbar.
