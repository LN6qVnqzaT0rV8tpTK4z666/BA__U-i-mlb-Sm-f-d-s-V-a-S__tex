\chapter{\abstractname}

When assessing the trustworthiness of mathematical models, quantifying their associated uncertainties is essential. This involves interpreting different uncertainty types, identifying their sources, selecting quantitative metrics, and applying computational methods for uncertainty estimation and solution analysis. Due to high computational demands, surrogate models are used to approximate complex real-world behaviour. Machine learning-based models offer a way to handle both aleatoric and epistemic uncertainty. This thesis demonstrates their use within a surrogate modelling framework for AUVs, employing Evidential Deep Learning for uncertainty analysis. The study shows how these uncertainties can be managed to assess their impact on aggregated uncertainty indicators. The results suggest that the validation of ML-based surrogate models benefits from such aggregated measures. \newline

\begin{otherlanguage}{ngerman}
Bei der Bewertung der Vertrauenswürdigkeit mathematischer Modelle ist die Quantifizierung ihrer Unsicherheiten entscheidend. Dies umfasst die Interpretation unterschiedlicher Unsicherheitsarten, die Identifikation ihrer Quellen, die Auswahl geeigneter Metriken sowie den Einsatz rechnergestützter Methoden zur Bestimmung und Analyse. Aufgrund hoher Rechenaufwände werden Surrogatmodelle zur Näherung komplexer Zusammenhänge eingesetzt. Modelle auf Basis maschinellen Lernens ermöglichen den Umgang mit aleatorischer und epistemischer Unsicherheit. In dieser Arbeit wird ihre Anwendung im Kontext von AUVs demonstriert, wobei Evidential Deep Learning zur Unsicherheitsanalyse verwendet wird. Die Untersuchung zeigt, wie sich diese Unsicherheiten steuern lassen und welchen Einfluss sie auf aggregierte Unsicherheitskennwerte haben. Die Ergebnisse deuten darauf hin, dass die Validierung ML-basierter Surrogatmodelle von aggregierten Unsicherheitsmaßen profitiert.
\end{otherlanguage}


%TODO: Abstract
% %\begin*{section}
% When analysing the trustworthiness of mathematical models and their practical application, quantifying the uncertainties associated with the model is relevant. This process requires the interpretation of different types of uncertainties, the identification of their sources, a choice of specific quantitative metrics, the development of engineering and computational methods to determine these metrics, as well as engineering and computational techniques to perform a solution analysis. Such an analysis is often based on an extensive parameter study to develop a solution to the model. In practice, this results in a computational intensity for relevant problems, which requires surrogate models to represent simplifications of complex behaviour in reality. Machine learning-based models offer a perspective for controlling both aleatoric and epistemic uncertainty. The goal is to show their application within a surrogate model framework for current use in AUVs. For this purpose, Evidential Deep Learning is used for uncertainty analysis. The analysis carried out shows how aleatoric and epistemic uncertainty can be controlled to demonstrate their influence on an aggregated uncertainty quantity of interest. The results suggest that the validation process of the machine learning-based surrogate model benefits from aggregated uncertainty indicators.
% %\end*{section}


% \begin{otherlanguage}{ngerman}
% %\begin*{section}
% Bei der Analyse der Vertrauenswürdigkeit von mathematischen Modellen und ihrer praktischen Anwendung ist die Quantifizierung der mit dem Modell verbundenen Unsicherheiten von Bedeutung. Dieser Prozess erfordert die Interpretation verschiedener Arten von Unsicherheiten, die Identifizierung ihrer Quellen, eine Auswahl spezifischer quantitativer Metriken, die Entwicklung von technischen und rechnerischen Methoden zur Bestimmung dieser Metriken sowie ingenieurtechnische und computergestützte Methoden zur Durchführung einer Lösungsanalyse. Eine solche Analyse stützt sich häufig auf eine umfangreiche Parameterstudie, um eine Lösung für das Modell zu entwickeln. In der Praxis führt dies zu einer hohen Rechenintensität für relevante Probleme, die Ersatzmodelle erfordert, um Vereinfachungen des komplexen Verhaltens in der Realität darzustellen. Auf maschinellem Lernen basierende Modelle bieten eine Perspektive für die Steuerung aleatorischer und epitemistischer Unsicherheit. Ziel ist es, die Anwendung im Rahmen eines Surrogatmodells in der aktuellen Anwendung für AUVs zu zeigen. Zu diesem Zweck wird Evidential Deep Learning zur Unsicherheitsanalyse eingesetzt. Die durchgeführte Analyse zeigt, wie aleatorische und epitemistische Unsicherheit gesteuert werden kann, um den Einfluss auf eine aggregierte Unsicherheitsgröße von Interesse zu zeigen. Die Ergebnisse legen nahe, dass der Validierungsprozess des maschinenlernbasierten Surrogatmodells von aggregierten Unsicherheitsindikatoren profitiert.
% %\end*{section}
% \end{otherlanguage}