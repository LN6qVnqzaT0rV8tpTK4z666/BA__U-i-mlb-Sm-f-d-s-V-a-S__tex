\newpage

% Apply custom geometry for a single page
\newgeometry{left=1cm,right=1cm,top=2cm,bottom=1.5cm}

\begin{formelsammlung}

  \section*{Formelzeichen}

  \begin{mdframed}[style=exercise]
    \begin{tabularx}{\textwidth}{>{$}l<{$} X}
        \toprule
        \textbf{Symbol} & \textbf{Bedeutung} \\
        \midrule
        \alpha_{1} & Signifikanzniveau in \gls{Conformal Prediction}, steuert die Breite der Konfidenzintervalle \\
        \alpha_{2} & Evidenzparameter (Shape) der evidentiellen Verteilung \\
        \beta & Skalenparameter der evidentiellen Verteilung \\
        \nu & Evidenzparameter in evidenzbasierten Netzen; steuert die Gewichtung zwischen epistemischer und aleatorischer Unsicherheit \\
        p_{\text{dropout}} & Dropout-Rate im \gls{Bayesianische neuronale Netze}, beeinflusst die epistemische Unsicherheit \\
        v & Varianzbezogener Parameter in Evidential Regression \\
        \mu & Erwartungswert der Vorhersage \\
        \hat{y} & Vorhersagewert des Modells \\
        y & Beobachteter (tatsächlicher) Wert \\
        \bar{\sigma}^2_{\text{total}} & Mittlere Gesamtvarianz (Summe aus epistemischer und aleatorischer Varianz) \\
        \bar{\sigma}^2_{\text{aleatoric}} & Mittlere aleatorische Varianz (Datenrauschen) \\
        \bar{\sigma}^2_{\text{epistemic}} & Mittlere epistemische Varianz (Modellunsicherheit) \\
        \sigma_{\text{aleatorisch}}^2 & Aleatorische Varianz (Datenrauschen) \\
        \sigma_{\text{epistemisch}}^2 & Epistemische Varianz (Modellunsicherheit) \\
        \sigma_{\text{total}}^2 & Gesamtvarianz \\
        \text{RMSE} & Root Mean Squared Error (Wurzel der mittleren quadratischen Abweichung) \\
        \text{MAE} & Mean Absolute Error (Mittlerer absoluter Fehler) \\
        \text{Coverage} & Abdeckungswahrscheinlichkeit der Unsicherheitsintervalle \\
        \text{Calibration Error} & Maß für die Diskrepanz zwischen prognostizierter Konfidenz und tatsächlicher Trefferquote \\
        \text{Mean Interval Width} & Mittlere Breite der berechneten Unsicherheitsintervalle \\
        \text{Score} & Aggregiertes Maß für die Gesamtperformance eines Modells \\
        \text{RUI} & Relative Uncertainty Index, relatives Maß für Unsicherheitsbreite im Verhältnis zum Fehlermaß \\
        \text{NLL} & Negative Log-Likelihood \\
        \text{KL}(P \,\|\, Q) & Kullback-Leibler-Divergenz zwischen Verteilungen \( P \) und \( Q \) \\
        \bottomrule
    \end{tabularx}
  \end{mdframed}

\end{formelsammlung}

% Restore the original geometry for subsequent pages
\restoregeometry
