\newpage

% Für eine Seite mit geändertem Rand:
\newgeometry{left=1cm,right=1cm,top=4cm,bottom=2.5cm}

\chapter*{Formelzeichenverzeichnis}
\addcontentsline{toc}{chapter}{Formelzeichenverzeichnis}

\begin{mdframed}[style=exercise]
\begin{tabularx}{\textwidth}{>{$}l<{$} X l}
\toprule
\textbf{Symbol} & \textbf{Bedeutung} & \textbf{Einheit} \\
\midrule
\alpha_{1} & Signifikanzniveau in \gls{Conformal Prediction}, steuert die Breite der Konfidenzintervalle & [–] \\
\alpha_{2} & Evidenzparameter (Shape) der evidentiellen Verteilung & [–] \\
\beta & Skalenparameter der evidentiellen Verteilung & [–] \\
\nu & Evidenzparameter in evidenzbasierten Netzen; steuert die Gewichtung zwischen epistemischer und aleatorischer Unsicherheit & [–] \\
p_{\text{dropout}} & Dropout-Rate im \gls{Bayesianische neuronale Netze}, beeinflusst die epistemische Unsicherheit & [–] \\
v & Varianzbezogener Parameter in Evidential Regression & [–] \\
\mu & Erwartungswert der Vorhersage & [wie Zielgröße] \\
\hat{y} & Vorhersagewert des Modells & [wie Zielgröße] \\
y & Beobachteter (tatsächlicher) Wert & [wie Zielgröße] \\
\bar{\sigma}^2_{\text{total}} & Mittlere Gesamtvarianz (Summe aus epistemischer und aleatorischer Varianz) & [(wie Zielgröße)$^2$] \\
\bar{\sigma}^2_{\text{aleatoric}} & Mittlere aleatorische Varianz (Datenrauschen) & [(wie Zielgröße)$^2$] \\
\bar{\sigma}^2_{\text{epistemic}} & Mittlere epistemische Varianz (Modellunsicherheit) & [(wie Zielgröße)$^2$] \\
\sigma_{\text{aleatorisch}}^2 & Aleatorische Varianz (Datenrauschen) & [(wie Zielgröße)$^2$] \\
\sigma_{\text{epistemisch}}^2 & Epistemische Varianz (Modellunsicherheit) & [(wie Zielgröße)$^2$] \\
\sigma_{\text{total}}^2 & Gesamtvarianz & [(wie Zielgröße)$^2$] \\
\text{RMSE} & Root Mean Squared Error (Wurzel der mittleren quadratischen Abweichung) & [wie Zielgröße] \\
\text{MAE} & Mean Absolute Error (Mittlerer absoluter Fehler) & [wie Zielgröße] \\
\text{Coverage} & Abdeckungswahrscheinlichkeit der Unsicherheitsintervalle & [–] \\
\text{Calibration Error} & Maß für die Diskrepanz zwischen prognostizierter Konfidenz und tatsächlicher Trefferquote & [–] \\
\text{Mean Interval Width} & Mittlere Breite der berechneten Unsicherheitsintervalle & [wie Zielgröße] \\
\text{Score} & Aggregiertes Maß für die Gesamtperformance eines Modells & [–] \\
\text{RUI} & Relative Uncertainty Index, relatives Maß für Unsicherheitsbreite im Verhältnis zum Fehlermaß & [–] \\
\text{NLL} & Negative Log-Likelihood & [–] \\
\text{KL}(P \,\|\, Q) & Kullback-Leibler-Divergenz zwischen Verteilungen \( P \) und \( Q \) & [–] \\
\bottomrule
\end{tabularx}
\end{mdframed}

\restoregeometry
