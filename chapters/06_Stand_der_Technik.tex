% !TeX root = ../main.tex

\chapter{Stand der Technik}
\label{chapter:stand-der-technik}



% R1 Welche auf maschinellem Lernen basierenden Surrogatmodelle sind für die Quantifizierung von Unsicherheiten geeignet?

% 

% GPR Gaußsche Prozessregression
% Deep Ensembles 
% Hybrid-Methoden, BAMOES
% BNNs Bayesianische neuronale Netze

% R2 Welche allgemeinen und modellspezifischen Faktoren beeinflussen das Lernen von Unsicherheiten in ML-Modellen?



% R3 Wie können diese Einflussfaktoren abgeschwächt werden, um die Unsicherheitsabschätzung zu verbessern?
% R4 Inwieweit kann ML-basierte Unsicherheitsquantifizierung zuverlässig den in realen Anwendungsszenarien beobachteten Unsicherheiten entsprechen?

% Die spezifischen Aufgaben dieser Arbeit umfassen:


% R5 Durchführung einer umfassenden Literaturübersicht über UQ-Techniken in der ML-basierten Surrogatmodellierung.
% R6 Identifizierung und Analyse geeigneter Unsicherheitsmetriken für Surrogatmodelle, wobei zwischen aleatorischer und epistemischer Unsicherheit unterschieden wird.
% R7 Vergleich modellspezifischer und anwendungsspezifischer Bewertungsmetriken für die Quantifizierung von Unsicherheit.
% R8 Untersuchung und Vergleich verschiedener UQ-Ansätze für ML-basierte Surrogatmodelle, insbesondere Bayes'sche neuronale Netze und konforme Vorhersagen.
% R9 Untersuchung des theoretischen Einflusses verschiedener Faktoren auf die Unsicherheitsschätzungen ausgewählter Modelle.
% R10 Implementierung und Validierung der entwickelten Unsicherheitsmetriken und Modellierungstechniken in einer realistischen Anwendung, wie z.\,B. der Unsicherheitsabschätzung in kritischen Regionen von Simulationsdaten.
% R11 Verfassen eines kurzen Jupyter Notebooks, das die wichtigsten Ansätze und Ergebnisse der Arbeit zusammenfasst.




% Datenqualität 

% ~\parencite{AndreasKreutz2022}
% ~\parencite{AndreasKreutz2022b}

% EDNN Grundlage 

% ~\parencite{Choi2017}

EDNN Regression 

~\parencite{AlexanderAmini2020}

EDNN Ensemble 

~\parencite{Schreck2023}

% Ninad

% ~\parencite{Gopakumar2024}

EDNN Entwicklung

~\parencite{Deng2023}



%  

\parencite{oberkampf2010}


% benchmark
% K. GREENMAN, A. SOLEIMANY, K. YANG: Benchmarking Uncertainty Quantification For Protein Engineering. URL https://openreview.net/pdf?id=G0vuqNwxaeA. – Aktualisierungsdatum: 21.04.2022 – Überprüfungsdatum 14.04.2025 



% \section{Autonome Unterwasserfahrzeuge (AUVs)}

% Der aktuelle Stand der Technik bei autonomen Unterwasserfahrzeugen (AUVs) zeigt einen zunehmenden Einsatz von \textbf{machine-learning-basierten Surrogatmodellen}, etwa zur Umgebungsmodellierung, Pfadplanung oder Systemdiagnose. 

% In komplexen maritimen Umgebungen sind AUVs typischerweise mit \emph{aleatorischer Unsicherheit} konfrontiert – beispielsweise durch verrauschte Sensordaten, Strömungseinflüsse oder schwankende Sichtverhältnisse. Surrogatmodelle wie \textit{Gaussian Processes}, \textit{Bayesian Neural Networks} oder \textit{Evidential Deep Learning} ermöglichen nicht nur schnelle Approximationen physikalischer Modelle, sondern auch die quantitative Erfassung \emph{epistemischer Unsicherheit} in bislang unkartierten Regionen.

% Moderne Entscheidungsverfahren berücksichtigen zunehmend unsicherheitsbewusste Vorhersagen, z.\,B.\ bei:

% \begin{itemize}
%   \item adaptiver Missionsplanung,
%   \item Risikoabschätzung in Echtzeit,
%   \item Bewertung vergrabener Munitionsfunde.
% \end{itemize}

% Hybride Konzepte aus lernbasierten Surrogaten und probabilistischer Modellierung gelten als vielversprechender Ansatz für robuste und erklärbare AUV-Systeme unter Unsicherheit~\parencite{cui2020, yan2021}.

% \section{Model Predictive Control (MPC) unter Unsicherheit}

% Aktuelle Arbeiten zur Model Predictive Control (MPC) zeigen, dass klassische deterministische Modelle zunehmend durch \textbf{stochastische und robuste Varianten} ergänzt werden. Moderne Verfahren wie \emph{Stochastic MPC} und \emph{Robust MPC} integrieren probabilistische Prädiktionsintervalle zur Modellierung \emph{aleatorischer Unsicherheit}.

% Zunehmend wird jedoch auch \textbf{epistemische Unsicherheit} berücksichtigt, insbesondere bei datengetriebenen Modellen wie \textit{Deep Neural Networks} oder \textit{Gaussian Processes}, deren Generalisierungsfähigkeit außerhalb des Trainingsbereichs limitiert ist.

% Ansätze wie \textit{Bayesian MPC} oder \textit{Evidential MPC} versuchen, beide Unsicherheitsarten zu modellieren und in Optimierungsstrategien einzubetten – z.\,B.\ mittels:

% \begin{itemize}
%   \item adaptiver Kostenfunktionen,
%   \item probabilistischer Constraints,
%   \item Szenariostreuung zur Absicherung.
% \end{itemize}

% Beispielhafte Open-Source-Projekte:

% \begin{itemize}
%   \item \url{https://github.com/do-mpc/do-mpc}
%   \item \url{https://github.com/lucasrm25/Gaussian-Process-based-Model-Predictive-Control}
%   \item \url{https://github.com/TinyMPC/TinyMPC}
% \end{itemize}

% \section{Surrogatmodelle und Unsicherheitsdarstellung}

% % Surrogatmodelle dienen der effizienten Approximation komplexer physikalischer Systeme. Typische Verfahren umfassen~\parencite{sudret2017, tik2025}:

% \begin{itemize}
%   \item \textbf{Kriging / Gaussian Process Regression (GPR)}
%   \item \textbf{Polynomial Chaos Expansion (PCE)}
%   \item \textbf{Gradient-Enhanced Kriging (GEK)}
%   \item \textbf{Radiale Basisfunktionen (RBF)}
%   \item \textbf{Support Vector Machines (SVM)}
% \end{itemize}

% Im Projektkontext kommen insbesondere \textbf{Bayessche Neuronale Netze (BNNs)} und \textbf{Ensemble-Methoden} zur Anwendung.

% \section{Meta-Unsicherheit und Unsicherheitsbewertung}

% % Der aktuelle Stand der Technik zur \textbf{Meta-Unsicherheit} bezieht sich auf die \emph{übergeordnete Aggregation und Bewertung} verschiedener Unsicherheitsquellen in Surrogatmodellen. Während klassische Methoden aleatorische und epistemische Unsicherheiten getrennt modellieren, zielen neuere Ansätze auf deren gemeinsame Bewertung ab~\parencite{schmitt2022}.

% Methoden wie:

% \begin{itemize}
%   \item \textit{Uncertainty Meta-Learning},
%   \item \textit{Uncertainty Calibration Layers},
%   \item \textit{Evidential Frameworks},
% \end{itemize}

% \noindent
% ermöglichen die adaptive Gewichtung von Unsicherheiten je nach Szenariokontext und Modellkonfidenz. Ziel ist eine robuste, vergleichbare und erklärbare Modellbewertung für sicherheitskritische Anwendungen.

% \section{Sensitivitätsanalyse und Kalibrierung}

% % Die Sensitivitätsanalyse quantifiziert den Einfluss einzelner Eingangsgrößen auf Modellvorhersagen oder Unsicherheiten. Sie hilft, dominante Einflussfaktoren systematisch zu identifizieren~\parencite{borgonovo2017, tunkiel2020}. Dabei unterscheidet man:

% \begin{itemize}
%   \item \textbf{Invasive Methoden:} direkt im Modell integriert,
%   \item \textbf{Nicht-invasive Methoden:} modellunabhängige Blackbox-Analyse.
% \end{itemize}

% Die \textbf{Modellkalibrierung} dient dazu, Vorhersagen an die Realität (Ground Truth) anzupassen, etwa durch Parameterabgleich oder Szenariobasierung.

% \section{Zeitreihenanalyse}

% % Für die Analyse zeitabhängiger Sensordaten – wie sie bei AUVs typischerweise auftreten – kommen moderne Zeitreihenanalyse-Tools zum Einsatz. Ein Beispiel dafür ist das \textit{TSFEL Framework}~\parencite{tsfel2025}:

% \begin{itemize}
%   \item \url{https://github.com/fraunhoferportugal/tsfel}
% \end{itemize}

% \section{Zusammenfassung}

% Der Stand der Technik zeigt, dass zwar zahlreiche Methoden zur Unsicherheitsquantifizierung existieren, ein systematischer, modellagnostischer und \textbf{integrierter Ansatz zur Szenariobewertung} unter Berücksichtigung beider Unsicherheitsarten jedoch fehlt. Diese Arbeit zielt darauf ab einen methodischen Beitrag zur transparenten Bewertung maschineller Lernverfahren in sicherheitskritischen Anwendungen zu leisten.


~\parencite{MatplotlibDevelopmentTeam.2024}
~\parencite{NumPyDevelopers.2024}
