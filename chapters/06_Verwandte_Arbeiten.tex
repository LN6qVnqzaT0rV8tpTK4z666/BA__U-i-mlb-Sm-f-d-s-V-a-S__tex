% !TeX root = ../main.tex

\chapter{Verwandte Arbeiten}\label{chapter:verwandte-arbeiten}


% \begin{otherlanguage}{american}

% \paragraph{Ulmer et al. (2023)}
% Ulmer et al.~\parencite{Ulmer2023} provide a comprehensive overview of evidential deep learning (EDL) and its use for uncertainty estimation. In particular, the authors distinguish between \emph{Prior Networks}, which model uncertainties by separately predicting distribution parameters, and \emph{Posterior Networks}, which are based on Bayesian approximation. The paper presents various loss functions, including evidential losses for regression and classification, and discusses advantages and disadvantages of EDL compared to classical Bayesian methods such as variational inference or deep ensembles. In addition, the survey highlights challenges such as calibration, interpretability and the practical applicability of EDL methods. Ulmer et al. thus offers a current state of research and a well-founded classification of evidence-based methods in the spectrum of uncertainty quantification. 

% \section{Definition of the work}

% The present work does not deal with the development of new machine learning architectures, but uses existing model structures Evidential Deep Learning specifically for uncertainty assessment in scenario-based surrogate models. No general system identification, validation or control is developed, but the focus is on the analysis, aggregation and interpretation of uncertainties, not on dynamic compensation.
% There is no exploration of complete control systems or classical MPC implementations, but a model consideration in the context of uncertain input space scenarios. Similarly, no physical AUV experiments or real-time systems are implemented, but simulations or model-based methods are considered. The work also distinguishes itself from pure fault diagnosis, classical statistical quality control, complete safety cases or normative certification methodology. (Furthermore, the aim is not to develop a complete method in the sense of a software framework, but to document an exemplary and research-oriented application).

% \end{otherlanguage}


\begin{otherlanguage}{ngerman}

\paragraph{Ulmer et al. (2023)}
Ulmer et al.~\parencite{Ulmer2023} liefern eine umfassende Übersicht über \gls{EvidentialDeepLearning} und dessen Einsatz zur Unsicherheitsabschätzung. Die Autoren unterscheiden insbesondere zwischen \emph{Prior Networks}, die Unsicherheiten durch separate Vorhersage von Verteilungsparametern modellieren, und \emph{Posterior Networks}, die auf bayesianischer Approximation basieren. Die Arbeit stellt verschiedene Loss-Funktionen vor, darunter Evidential Losses für Regression und Klassifikation, und diskutiert Vor- und Nachteile von \gls{EvidentialDeepLearning} gegenüber klassischen bayesianischen Verfahren wie \gls{variationalinference} oder Deep Ensembles. Darüber hinaus beleuchtet der Survey Herausforderungen wie Kalibrierung, Interpretierbarkeit und die praktische Anwendbarkeit von \gls{EvidentialDeepLearning}-Methoden. Damit bietet Ulmer et al. einen aktuellen Stand der Forschung und eine fundierte Einordnung evidenzbasierter Verfahren in das Spektrum der Unsicherheitsquantifizierung. 

\section{Abgrenzung der Arbeit}

Die vorliegende Arbeit behandelt nicht die Entwicklung neuer \gls{machinelearning}-Architekturen, sondern setzt bestehende Modellstrukturen \gls{EvidentialDeepLearning} gezielt zur Unsicherheitsbewertung in szenariobasierten Surrogatmodellen ein. Es wird keine generelle Systemidentifikation, Validierung oder Regelung entwickelt, sondern der Fokus liegt auf der Analyse, Aggregation und Interpretation von Unsicherheiten, nicht auf der dynamischen Kompensation.
Es erfolgt keine Exploration vollständiger Regelungssysteme oder klassischer \gls{ModelPredictiveControl}-Implementierungen, sondern eine Modellbetrachtung im Kontext unsicherer Eingaberaum-Szenarien. Ebenso werden keine physikalischen \gls{Autonomous Underwater Vehicle}-Experimente oder Realzeitsysteme implementiert, sondern Simulationen bzw. modellbasierte Methoden betrachtet. Die Arbeit grenzt sich außerdem ab von reiner Fehlerdiagnose, klassischer statistischer Qualitätskontrolle, vollständigen Safety-Cases oder normativer Zertifizierungsmethodik. (Ferner wird keine vollständige Methodenentwicklung im Sinne eines Software-Frameworks angestrebt, sondern eine exemplarische und forschungsnahe Anwendung dokumentiert.)

\end{otherlanguage}
