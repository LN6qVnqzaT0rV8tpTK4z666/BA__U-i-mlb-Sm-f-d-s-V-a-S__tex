% !TeX root = ../main.tex
% Add the above to each chapter to make compiling the PDF easier in some editors.

\chapter{Einleitung}
\label{chapter:introduction}



\begin{otherlanguage}{american}
% \chapter{Introduction}
% \label{chapter:introduction}

% \section{Introduction}

% This bachelor thesis deals with the question of how uncertainties in machine learning-based surrogate models can be quantified and made comparable in the context of scenario-based validation of autonomous systems. The topic is of high technical relevance for the verification of complex systems and is also considered worthy of support from a societal perspective.

% \section{Societal relevance}

% In collaboration with the German Aerospace Center (DLR), the Federal Ministry of Education and Research (BMBF) emphasizes the high importance of trustworthy and explainable AI systems in its funding strategy for machine learning ~\parencite{bmbf2025}. In safety-critical areas such as maritime robotics, reliable uncertainty assessment is essential for the use of autonomous systems.

% \section{Practical background}

% The work in the project is based on an industrial internship from October to December 2024. As part of the bachelor's degree program in Systems Engineering, the project \enquote{Development of scenario-based validation methods for autonomous maritime systems} was participated in. The insights gained there form the starting point for the scientific treatment of the present issue.

% \section{Integration into practice}

% Atlas Elektronik GmbH has been active as a technology leader in the maritime sector for over 130 years. With a view to future applications, the company's website emphasizes that systems should not only be functional, but also intelligently networked across platforms. The company's own research institute, A-Lab, serves as an interface between academic research and industrial development.

% \section{Need for a solution and application context}

% A concrete need arises from the report by the German Alliance for Marine Research (DAM) on the removal of contaminated sites in the North Sea and Baltic Sea ~\parencite{dam2024}. Projects by Atlas Elektronik GmbH aim to use autonomous underwater vehicles (AUVs) for the detection and removal of munitions remnants.
\end{otherlanguage}



\begin{otherlanguage}{ngerman}
\section{Einführung}

Diese Bachelorarbeit beschäftigt sich mit der Frage, wie Unsicherheiten in machine-learning-basierten Surrogatmodellen im Kontext der szenariobasierten Validierung autonomer Systeme quantifiziert und vergleichbar gemacht werden können. Die Thematik besitzt eine hohe technische Relevanz für die Verifikation komplexer Systeme und wird auch aus gesellschaftlicher Perspektive als förderwürdig eingestuft.

\section{Gesellschaftliche Relevanz}

Das \gls{Bundesministerium für Bildung und Forschung} hebt in Zusammenarbeit mit dem \gls{Deutschen Zentrum für Luft- und Raumfahrt} in seiner Förderstrategie zum Maschinellen Lernen die hohe Bedeutung vertrauenswürdiger und erklärbarer KI-Systeme hervor (vgl. Bundesministerium für Bildung und Forschung ~\parencite{bmbf2025}). In sicherheitskritischen Bereichen wie der maritimen Robotik ist eine verlässliche Unsicherheitsabschätzung essenziell für den Einsatz autonomer Systeme.

\section{Praktischer Hintergrund}

Als Grundlage für die Beteiligung im Projekt dient ein Industriepraktikum im Zeitraum Oktober bis Dezember 2024. Im Rahmen des Bachelorstudiengangs Systems Engineering wurde im Projekt \enquote{Entwicklung szenariobasierter Validierungsmethoden für autonome maritime Systeme} (VaMai) mitgewirkt. 

\section{Einbindung in die Praxis}

Die Atlas Elektronik GmbH ist mit einer über 130-jährigen Geschichte als Technologieträger im maritimen Bereich tätig. Mit Blick auf zukünftige Anwendungen wird auf der Unternehmenswebseite betont, dass Systeme nicht nur funktional, sondern auch plattformübergreifend intelligent vernetzt werden sollen. Das firmeneigene Forschungsinstitut A-Lab dient hierbei als Schnittstelle zwischen akademischer Forschung und industrieller Entwicklung.

\section{Lösungsbedarf und Anwendungskontext}

Ein konkreter Bedarf ergibt sich aus dem Bericht der \gls{Deutschen Allianz für Meeresforschung} zur Altlastenbeseitigung in Nord- und Ostsee (vgl. Deutsche Allianz für Meeresforschung ~\parencite{dam2024}). Projekte der Atlas Elektronik GmbH zielen darauf ab, autonome Unterwasserfahrzeuge (\gls{Autonomous Underwater Vehicle}s) für die Detektion und Entfernung von Munitionsresten einzusetzen. 
\end{otherlanguage}
