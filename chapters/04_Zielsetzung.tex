% !TeX root = ../main.tex

\chapter{Zielsetzung}
\label{chapter:zielsetzung}



\begin{otherlanguage}{american}
% \chapter{Objective}
% \label{chapter:objective}

% \section{Objective of the thesis}

% The objective of this thesis is to develop and analyze a framework for the quantitative assessment and aggregation of uncertainties in machine learning processes for surrogate models of autonomous systems. The aim is to systematically distinguish between aleatory and epistemic uncertainties, quantify them, and validate them using engineering methods.

% Particular attention is paid to the conception of a meta-uncertainty that allows conclusions to be drawn about the reliability of different model approaches within the framework of scenario-based validation. The aim is to contribute to the methodological foundation of uncertainty analysis in safety-critical applications.

% \section{Research Questions}

% The following research questions (RQ) form the central content framework of the thesis. Each question is formulated in such a way that it can be addressed practically and theoretically using suitable methods (e.g., simulation, ML training, uncertainty modeling). 

% The central research questions of this thesis are:

% \begin{itemize}
%   \item {R1} Which machine learning-based surrogate models are suitable for quantifying uncertainties?
%   \item {R2} Which general and model-specific factors influence the learning of uncertainties in ML models?
%   \item {R3} How can these influencing factors be mitigated to improve uncertainty estimation?
%   \item {R4} To what extent can ML-based uncertainty quantification reliably correspond to the uncertainties observed in real-world application scenarios?
% \end{itemize}

% The specific tasks of this work include:

% \begin{itemize}
%   \item {R5} Conducting a comprehensive literature review of UQ techniques in ML-based surrogate modeling.
%   \item {R6} Identifying and analyzing suitable uncertainty metrics for surrogate models, distinguishing between aleatory and epistemic uncertainty.
%   \item {R7} Comparing model-specific and application-specific evaluation metrics for quantifying uncertainty.
%   \item {R8} Investigating and comparing different UQ approaches for ML-based surrogate models, in particular Bayesian neural networks and conformal predictions.
%   \item {R9} Investigation of the theoretical influence of various factors on the uncertainty estimates of selected models.
%   \item {R10} Implementation and validation of the developed uncertainty metrics and modeling techniques in a realistic application, such as uncertainty estimation in critical regions of simulation data.
%   \item {R11} Writing a short Jupyter Notebook summarizing the most important approaches and results of the work.
% \end{itemize}
\end{otherlanguage}



\begin{otherlanguage}{ngerman}
\section{Zielsetzung der Arbeit}

Ziel dieser Arbeit ist die Entwicklung und Analyse eines Frameworks zur quantitativen Bewertung und Untersuchung von \gls{uncertaintyquantification} in \gls{machinelearning}-Verfahren für Surrogatmodelle autonomer Systeme. Dabei sollen sowohl \gls{Aleatorische Unsicherheit} als auch \gls{Epistemische Unsicherheit} systematisch unterschieden, quantifiziert und mit ingenieurwissenschaftlichen Methoden validiert werden.

Ein besonderes Augenmerk liegt auf der Konzeption einer Meta-Unsicherheit, die Rückschlüsse auf die Vertrauenswürdigkeit unterschiedlicher Modellansätze im Rahmen szenariobasierter Validierung zulässt. Damit soll ein Beitrag zur methodischen Fundierung der Unsicherheitsanalyse in sicherheitskritischen Anwendungen geleistet werden.

\section{Forschungsfragen}

Die folgenden Forschungsfragen (Research Questions, $RQ$) bilden den zentralen inhaltlichen Rahmen der Arbeit. Jede Frage ist so formuliert, dass sie durch geeignete Methoden, zum Beispiel Simulation, \gls{machinelearning}-Training, Unsicherheitsmodellierung, praktisch und theoretisch bearbeitet werden kann. 

Die zentralen Forschungsfragen dieser Arbeit lauten:

\begin{itemize}
  \item \textbf{RQ1} \textit{Welche auf maschinellem Lernen basierenden Surrogatmodelle sind für die Quantifizierung von Unsicherheiten geeignet?}
  \item \textbf{RQ2} \textit{Welche allgemeinen und modellspezifischen Faktoren beeinflussen das Lernen von Unsicherheiten in \gls{machinelearning}-Modellen?}
  \item \textbf{RQ3} \textit{Wie können diese Einflussfaktoren abgeschwächt werden, um die Unsicherheitsabschätzung zu verbessern?}
  \item \textbf{RQ4} \textit{Inwieweit kann \gls{machinelearning}-basierte Unsicherheitsquantifizierung zuverlässig den in realen Anwendungsszenarien beobachteten Unsicherheiten entsprechen?}
\end{itemize}

Die spezifischen Aufgaben dieser Arbeit umfassen:

\begin{itemize}
  \item \textbf{A1} \textit{Durchführung einer umfassenden Literaturübersicht über UQ-Techniken in der \gls{machinelearning}-basierten Surrogatmodellierung.}
  \item \textbf{A2} \textit{Identifizierung und Analyse geeigneter Unsicherheitsmetriken für Surrogatmodelle, wobei zwischen aleatorischer und epistemischer Unsicherheit unterschieden wird.}
  \item \textbf{A3} \textit{Vergleich modellspezifischer und anwendungsspezifischer Bewertungsmetriken für die Quantifizierung von Unsicherheit.}
  \item \textbf{A4} \textit{Untersuchung und Vergleich verschiedener UQ-Ansätze für \gls{machinelearning}-basierte Surrogatmodelle, insbesondere Bayes'sche neuronale Netze und konforme Vorhersagen.}
  \item \textbf{A5} \textit{Untersuchung des theoretischen Einflusses verschiedener Faktoren auf die Unsicherheitsschätzungen ausgewählter Modelle.}
  \item \textbf{A6} \textit{Implementierung und Validierung der entwickelten Unsicherheitsmetriken und Modellierungstechniken in einer realistischen Anwendung, wie z.\,B. der Unsicherheitsabschätzung in kritischen Regionen von Simulationsdaten.}
  \item \textbf{A7} \textit{Verfassen eines kurzen Jupyter Notebooks, das die wichtigsten Ansätze und Ergebnisse der Arbeit zusammenfasst.}
\end{itemize}
\end{otherlanguage}
