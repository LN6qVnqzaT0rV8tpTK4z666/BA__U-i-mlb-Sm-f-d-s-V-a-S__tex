% !TeX root = ../main.tex

\chapter{Zielsetzung}
\label{chapter:zielsetzung}

\section{Zielsetzung der Arbeit}

Ziel dieser Arbeit ist die Entwicklung und Analyse eines Frameworks zur quantitativen Bewertung und Aggregation von Unsicherheiten in maschinellen Lernverfahren für Surrogatmodelle autonomer Systeme. Dabei sollen sowohl aleatorische als auch epistemische Unsicherheiten systematisch unterschieden, quantifiziert und mit ingenieurwissenschaftlichen Methoden validiert werden.

Ein besonderes Augenmerk liegt auf der Konzeption einer Meta-Unsicherheit, die Rückschlüsse auf die Vertrauenswürdigkeit unterschiedlicher Modellansätze im Rahmen szenariobasierter Validierung zulässt. Damit soll ein Beitrag zur methodischen Fundierung der Unsicherheitsanalyse in sicherheitskritischen Anwendungen geleistet werden.

\section{Forschungsfragen}

Die folgenden Forschungsfragen (Research Questions, RQ) bilden den zentralen inhaltlichen Rahmen der Arbeit. Jede Frage ist so formuliert, dass sie durch geeignete Methoden (z.~B. Simulation, ML-Training, Unsicherheitsmodellierung) praktisch und theoretisch bearbeitet werden kann. 

Die zentralen Forschungsfragen dieser Arbeit lauten:

\begin{itemize}
  \item {R1} Welche auf maschinellem Lernen basierenden Surrogatmodelle sind für die Quantifizierung von Unsicherheiten geeignet?
  \item {R2} Welche allgemeinen und modellspezifischen Faktoren beeinflussen das Lernen von Unsicherheiten in ML-Modellen?
  \item {R3} Wie können diese Einflussfaktoren abgeschwächt werden, um die Unsicherheitsabschätzung zu verbessern?
  \item {R4} Inwieweit kann ML-basierte Unsicherheitsquantifizierung zuverlässig den in realen Anwendungsszenarien beobachteten Unsicherheiten entsprechen?
\end{itemize}

Die spezifischen Aufgaben dieser Arbeit umfassen:

\begin{itemize}
  \item {R5} Durchführung einer umfassenden Literaturübersicht über UQ-Techniken in der ML-basierten Surrogatmodellierung.
  \item {R6} Identifizierung und Analyse geeigneter Unsicherheitsmetriken für Surrogatmodelle, wobei zwischen aleatorischer und epistemischer Unsicherheit unterschieden wird.
  \item {R7} Vergleich modellspezifischer und anwendungsspezifischer Bewertungsmetriken für die Quantifizierung von Unsicherheit.
  \item {R8} Untersuchung und Vergleich verschiedener UQ-Ansätze für ML-basierte Surrogatmodelle, insbesondere Bayes'sche neuronale Netze und konforme Vorhersagen.
  \item {R9} Untersuchung des theoretischen Einflusses verschiedener Faktoren auf die Unsicherheitsschätzungen ausgewählter Modelle.
  \item {R10} Implementierung und Validierung der entwickelten Unsicherheitsmetriken und Modellierungstechniken in einer realistischen Anwendung, wie z.\,B. der Unsicherheitsabschätzung in kritischen Regionen von Simulationsdaten.
  \item {R11} Verfassen eines kurzen Jupyter Notebooks, das die wichtigsten Ansätze und Ergebnisse der Arbeit zusammenfasst.
\end{itemize}



% \chapter{Objective}
% \label{chapter:objective}

% \section{Objective of the thesis}

% The objective of this thesis is to develop and analyze a framework for the quantitative assessment and aggregation of uncertainties in machine learning processes for surrogate models of autonomous systems. The aim is to systematically distinguish between aleatory and epistemic uncertainties, quantify them, and validate them using engineering methods.

% Particular attention is paid to the conception of a meta-uncertainty that allows conclusions to be drawn about the reliability of different model approaches within the framework of scenario-based validation. The aim is to contribute to the methodological foundation of uncertainty analysis in safety-critical applications.

% \section{Research Questions}

% The following research questions (RQ) form the central content framework of the thesis. Each question is formulated in such a way that it can be addressed practically and theoretically using suitable methods (e.g., simulation, ML training, uncertainty modeling). 

% The central research questions of this thesis are:

% \begin{itemize}
%   \item {R1} Which machine learning-based surrogate models are suitable for quantifying uncertainties?
%   \item {R2} Which general and model-specific factors influence the learning of uncertainties in ML models?
%   \item {R3} How can these influencing factors be mitigated to improve uncertainty estimation?
%   \item {R4} To what extent can ML-based uncertainty quantification reliably correspond to the uncertainties observed in real-world application scenarios?
% \end{itemize}

% The specific tasks of this work include:

% \begin{itemize}
%   \item {R5} Conducting a comprehensive literature review of UQ techniques in ML-based surrogate modeling.
%   \item {R6} Identifying and analyzing suitable uncertainty metrics for surrogate models, distinguishing between aleatory and epistemic uncertainty.
%   \item {R7} Comparing model-specific and application-specific evaluation metrics for quantifying uncertainty.
%   \item {R8} Investigating and comparing different UQ approaches for ML-based surrogate models, in particular Bayesian neural networks and conformal predictions.
%   \item {R9} Investigation of the theoretical influence of various factors on the uncertainty estimates of selected models.
%   \item {R10} Implementation and validation of the developed uncertainty metrics and modeling techniques in a realistic application, such as uncertainty estimation in critical regions of simulation data.
%   \item {R11} Writing a short Jupyter Notebook summarizing the most important approaches and results of the work.
% \end{itemize}



% Die zentralen Forschungsfragen dieser Arbeit lauten:

% \begin{itemize}
%   \item Welche auf maschinellem Lernen basierenden Surrogatmodelle sind für die Quantifizierung von Unsicherheiten geeignet?
%   \item Welche allgemeinen und modellspezifischen Faktoren beeinflussen das Lernen von Unsicherheiten in ML-Modellen?
%   \item Wie können diese Einflussfaktoren abgeschwächt werden, um die Unsicherheitsabschätzung zu verbessern?
%   \item Inwieweit kann ML-basierte Unsicherheitsquantifizierung zuverlässig den in realen Anwendungsszenarien beobachteten Unsicherheiten entsprechen?
% \end{itemize}

% Die spezifischen Aufgaben dieser Arbeit umfassen:

% \begin{itemize}
%   \item Durchführung einer umfassenden Literaturübersicht über UQ-Techniken in der ML-basierten Surrogatmodellierung.
%   \item Identifizierung und Analyse geeigneter Unsicherheitsmetriken für Surrogatmodelle, wobei zwischen aleatorischer und epistemischer Unsicherheit unterschieden wird.
%   \item Vergleich modellspezifischer und anwendungsspezifischer Bewertungsmetriken für die Quantifizierung von Unsicherheit.
%   \item Untersuchung und Vergleich verschiedener UQ-Ansätze für ML-basierte Surrogatmodelle, insbesondere Bayes'sche neuronale Netze und konforme Vorhersagen.
%   \item Untersuchung des theoretischen Einflusses verschiedener Faktoren auf die Unsicherheitsschätzungen ausgewählter Modelle.
%   \item Implementierung und Validierung der entwickelten Unsicherheitsmetriken und Modellierungstechniken in einer realistischen Anwendung, wie z.\,B. der Unsicherheitsabschätzung in kritischen Regionen von Simulationsdaten.
%   \item Verfassen eines kurzen Jupyter Notebooks, das die wichtigsten Ansätze und Ergebnisse der Arbeit zusammenfasst.
% \end{itemize}


% \subsection*{RQ1: Quantifizierung aleatorischer und epistemischer Unsicherheiten}

% \textbf{Wie lassen sich aleatorische und epistemische Unsicherheiten in machine-learning-basierten Surrogatmodellen getrennt und vergleichbar quantifizieren?}

% \emph{Begründung:} Eine systematische Trennung und vergleichbare Quantifizierung beider Unsicherheitsarten unter einheitlichen Modellbedingungen fehlt bisher weitgehend in der Literatur.

% \vspace{1em}

% \subsection*{RQ2: Kritikalitätsraum-Definition}

% \textbf{Wie kann ein Kritikalitätsraum $\Omega_k$ formal definiert werden, um sicherheitsrelevante Szenarien basierend auf aggregierter Unsicherheit zu identifizieren?}

% \emph{Begründung:} Der Begriff Kritikalitätsraum wird bislang häufig informell verwendet. Eine formale, quantifizierbare Definition könnte die Aussagekraft szenariobasierter Validierung deutlich erhöhen.

% \vspace{1em}

% \subsection*{RQ3: Aggregationsmethoden für Unsicherheiten}

% \textbf{Welche Aggregationsmethoden eignen sich zur Verknüpfung lokal berechneter Unsicherheiten zu einem szenario- oder trajektorienbezogenen Vertrauensmaß?}

% \emph{Begründung:} Während lokale Unsicherheiten gängig sind, fehlt es an standardisierten Verfahren zur Aggregation in ein übergeordnetes Vertrauensmaß.

% \vspace{1em}

% \subsection*{RQ4 (optional): Einfluss des Signal-Rausch-Verhältnisses}

% \textbf{Wie wirkt sich das Signal-Rausch-Verhältnis auf die detektierbare Struktur kritischer Regionen im Surrogatmodell aus?}

% \emph{Begründung:} Die Empfindlichkeit gegenüber Unsicherheitssignalen hängt stark von der Rauschstruktur und Stichprobendichte ab – ein bislang wenig untersuchter Zusammenhang.

% \vspace{1em}

% \subsection*{RQ5 (optional, übergreifend): Meta-Unsicherheit aus Evidenz und Struktur}

% \textbf{Inwiefern kann die Kombination evidenzbasierter Modelle mit strukturellen Metriken (z.~B. Autokorrelation, Jensen-Shannon-Divergenz) zu einer erklärbaren Meta-Unsicherheit führen?}

% \emph{Begründung:} Meta-Unsicherheit ist ein neuartiges Konzept. Die Integration unterschiedlicher Metriken in ein übergeordnetes Maß ist bislang methodisch nicht etabliert.

% \vspace{2em}



% % Falls du diese verworfenen Fragen als Fußnote, Anhang oder Kommentar erhalten willst:
% % \begin{comment}
% % --- Verworfen ---
% % \begin{itemize}
% %   \item Entwicklung einer modellagnostischen Meta-Unsicherheit für bayessche ML-Modelle zur Validierung im Kritikalitätsraum.
% %   \item Systematische Differenzierung aleatorischer vs. epistemischer Unsicherheit in zeitparametrisierten Szenarien.
% %   \item Evaluation von Unsicherheitsmetriken (z. B. RMSE, ECE, Sobol) zur Validierung von Surrogatmodellen.
% % \end{itemize}
% % \end{comment}
