% !TeX root = ../main.tex

\chapter{Durchführung}\label{chapter:durchfuehrung}



\begin{otherlanguage}{american}
  
\end{otherlanguage}



\begin{otherlanguage}{ngerman}

In der Durchführung soll \textit{R10 Implementierung und Validierung der entwickelten Unsicherheitsmetriken
und Modellierungstechniken in einer realistischen Anwendung, wie z. B. der
Unsicherheitsabschätzung in kritischen Regionen von Simulationsdaten} beantwortet werden. \parencite{amini2020deep} liefert einen bekannten validierten Ansatz aus öffentlicher Forschung. Hergert 2025 liefert einen bekannten validierten Ansatz von Unternehmensseite. 

\section*{Bekannte Ansätze}

Zur Validierung der eingesetzten Methodik für \gls{Evidenzbasierte neuronale Netze} wird der von \parencite{amini2020deep} veröffentlichte Benchmark reproduziert. Hierfür wird die im Anhang dargestellte Aufwandsschätzung berücksichtigt, die auf die im Unternehmen verfügbare Hardware abgestimmt ist, (~\ref{sec:aufwand-edl}).

\paragraph{Vergleich Amini EDNN Regression} \parencite{amini2020deep} Sektion 4 Experiment liefert Ergebnisse für \gls{Evidenzbasierte neuronale Netze}.  
\begin{noindentquote}
\scriptsize
$
\{ \text{RMSE}, \text{NLL} \} 
\times 
\{ \text{dropout}, \text{ensembles}, \text{evidential} \} 
\times 
\{ \text{boston}, \text{concrete}, \text{energy}, \text{kin8nm}, \text{naval}, \text{power}, \text{protein}, \text{wine}, \text{yacht} \}
$  
\end{noindentquote}
Der Benchmark konnte in erster Konfiguration annähernd reproduziert werden. 
\begin{noindentquote}
  \footnotesize
  \texttt{python3 neurips2020/run\_uci\_dataset\_tests.py --datasets boston concrete energy-efficiency kin8nm naval power-plant protein wine yacht --num-trials 1 --num-epochs 5}
\end{noindentquote}
 mit Laufzeit von 9 Minuten 10.2 Sekunden $\leq$ 10 Minuten als \glqq Proof of concept\grqq. Die Konvergenz zu Aminis Werten ($\Delta (\text{RMSE}) = \left[ -0.5;\ -0.6 \right]$; $\Delta \text{NLL} = -0.08$) wird deutlich für die Konfiguration hinreichend umfangreicher Datensätze $\{\text{kin8nm}, \text{naval}, \text{protein} \}$. ~\ref{tab:comparison_ours_vs_amini_tabularx_v2} unter der Hardware-Limitierung aus Tabelle 1, Gerät 1, Hardware-Spezifikation, Anhang.

\begin{table*}[!htbp]
\centering
\scriptsize
\begin{tabularx}{\textwidth}{|l|X|X|X|X|X|X|}
\hline
\multirow{}{}{\textbf{Datensatz}} 
& \multicolumn{3}{c|}{\textbf{RMSE}} 
& \multicolumn{3}{c|}{\textbf{NLL}} \\ \cline{2-7}
& \textbf{Dropout} & \textbf{Ensemble} & \textbf{Evidential} 
& \textbf{Dropout} & \textbf{Ensemble} & \textbf{Evidential} \\ \hline
boston 
& 3.678 \newline \textit{2.896} \newline \mbox{$\Delta=+0.782$} 
& 4.956 \newline \textit{2.998} \newline \mbox{$\Delta=+1.958$} 
& 3.672 \newline \textit{2.886} \newline \mbox{$\Delta=+0.786$} 
& 4.636 \newline \textit{2.755} \newline \mbox{$\Delta=+1.881$} 
& 2.981 \newline \textit{2.838} \newline \mbox{$\Delta=+0.143$} 
& 2.621 \newline \textit{2.701} \newline \mbox{$\Delta=-0.080$} \\ \hline
concrete 
& 6.905 \newline \textit{5.367} \newline \mbox{$\Delta=+1.538$} 
& 6.705 \newline \textit{5.263} \newline \mbox{$\Delta=+1.442$} 
& 6.496 \newline \textit{5.163} \newline \mbox{$\Delta=+1.333$} 
& 5.254 \newline \textit{3.213} \newline \mbox{$\Delta=+2.041$} 
& 3.271 \newline \textit{3.193} \newline \mbox{$\Delta=+0.078$} 
& 3.277 \newline \textit{3.138} \newline \mbox{$\Delta=+0.139$} \\ \hline
energy-efficiency 
& 2.721 \newline \textit{1.200} \newline \mbox{$\Delta=+1.521$} 
& 2.516 \newline \textit{0.987} \newline \mbox{$\Delta=+1.529$} 
& 2.487 \newline \textit{0.984} \newline \mbox{$\Delta=+1.503$} 
& 4.699 \newline \textit{-0.964} \newline \mbox{$\Delta=+5.663$} 
& 1.920 \newline \textit{-1.315} \newline \mbox{$\Delta=+3.235$} 
& 1.879 \newline \textit{-1.376} \newline \mbox{$\Delta=+3.255$} \\ \hline
kin8nm 
& 0.095 \newline \textit{0.096} \newline \mbox{$\Delta=-0.001$} 
& 0.084 \newline \textit{0.080} \newline \mbox{$\Delta=+0.004$} 
& 0.088 \newline \textit{0.077} \newline \mbox{$\Delta=+0.011$} 
& 1.114 \newline \textit{-1.213} \newline \mbox{$\Delta=+2.327$} 
& -1.113 \newline \textit{-1.496} \newline \mbox{$\Delta=+0.383$} 
& -0.992 \newline \textit{-1.537} \newline \mbox{$\Delta=+0.545$} \\ \hline
naval 
& 0.005 \newline \textit{0.003} \newline \mbox{$\Delta=+0.002$} 
& 0.005 \newline \textit{0.002} \newline \mbox{$\Delta=+0.003$} 
& 0.005 \newline \textit{0.002} \newline \mbox{$\Delta=+0.003$} 
& -2.438 \newline \textit{-6.233} \newline \mbox{$\Delta=+3.795$} 
& -3.931 \newline \textit{-6.967} \newline \mbox{$\Delta=+3.036$} 
& -3.747 \newline \textit{-6.987} \newline \mbox{$\Delta=+3.240$} \\ \hline
power-plant 
& 3.349 \newline \textit{4.049} \newline \mbox{$\Delta=-0.700$} 
& 3.467 \newline \textit{4.021} \newline \mbox{$\Delta=-0.554$} 
& 3.377 \newline \textit{3.987} \newline \mbox{$\Delta=-0.610$} 
& 5.281 \newline \textit{2.795} \newline \mbox{$\Delta=+2.486$} 
& 2.678 \newline \textit{2.765} \newline \mbox{$\Delta=-0.087$} 
& 2.671 \newline \textit{2.755} \newline \mbox{$\Delta=-0.084$} \\ \hline
protein 
& 4.355 \newline \textit{4.512} \newline \mbox{$\Delta=-0.157$} 
& 3.939 \newline \textit{4.480} \newline \mbox{$\Delta=-0.541$} 
& 4.169 \newline \textit{4.430} \newline \mbox{$\Delta=-0.261$} 
& 4.266 \newline \textit{2.871} \newline \mbox{$\Delta=+1.395$} 
& 2.742 \newline \textit{2.837} \newline \mbox{$\Delta=-0.095$} 
& 2.706 \newline \textit{2.779} \newline \mbox{$\Delta=-0.073$} \\ \hline
wine 
& 0.826 \newline \textit{0.648} \newline \mbox{$\Delta=+0.178$} 
& 0.829 \newline \textit{0.631} \newline \mbox{$\Delta=+0.198$} 
& 0.722 \newline \textit{0.609} \newline \mbox{$\Delta=+0.113$} 
& 2.255 \newline \textit{-0.574} \newline \mbox{$\Delta=+2.829$} 
& 1.390 \newline \textit{-0.618} \newline \mbox{$\Delta=+2.008$} 
& 1.187 \newline \textit{-0.679} \newline \mbox{$\Delta=+1.866$} \\ \hline
yacht 
& 7.046 \newline \textit{1.711} \newline \mbox{$\Delta=+5.335$} 
& 9.197 \newline \textit{1.337} \newline \mbox{$\Delta=+7.860$} 
& 5.557 \newline \textit{1.252} \newline \mbox{$\Delta=+4.305$} 
& 5.170 \newline \textit{-1.637} \newline \mbox{$\Delta=+6.807$} 
& 3.041 \newline \textit{-2.257} \newline \mbox{$\Delta=+5.298$} 
& 2.401 \newline \textit{-2.341} \newline \mbox{$\Delta=+4.742$} \\ \hline
\end{tabularx}
\caption{Vergleich reproduzierte Ergebnisse mit Benchmark \textit{Amini et al. (2020)}}
\label{tab:comparison_ours_vs_amini_tabularx_v3}
\end{table*}


\paragraph{Vergleich Amini EDNN Klassifikation} \parencite{amini2020deep} \texttt{gen\_depth\_results.py} liefert Ergebnisse für \gls{Evidenzbasierte neuronale Netze} für die Aufgabe \glqq Monocular depth estimation\grqq. Der Benchmark konnte reproduziert werden mit einer Konfiguration $N_\text{batch}=25$, $N_\text{adv}=3$ unter der Hardware-Limitierung aus Tabelle 1, Gerät 1, Hardware-Spezifikation, Anhang.



\section*{Eigenentwicklung}

Benötigt wurde eine Datenpipeline. In Entwicklung mit Herr M.Sc. N. Mavani sollte für den Vergleich von \gls{Bayesianische neuronale Netze}, \gls{Evidenzbasierte neuronale Netze} als Baseline von Mavanis Seite auf \parencite{Depeweg2019}, von meiner Seite auf \parencite{Ulmer2023} aufgesetzt werden.\newline

Dazu wurden von \parencite{Depeweg2019} verwendete Datensätze verwendet und ergänzt um in \hyperref[sec:datensaetze]{Datensätze} gelistete. Aufgesetzte Stufen der Datenpipeline sind eine Vorverarbeitung zum Umwandeln von Formaten, Einlesen in Torch Dataset Klassen, Torch Trainingsklassen, Visualisierungsklassen.\newline

Aus den Datensätzen wurden Trainings und Visualisierungen abgeleitet und generifiziert. Ebenso wurden recherchierte Loss-Funktionen generifiziert. Trainings wurden auf einen modernen Standard ergänzt wie Checkpoints, Patience-Epochen, vergleichbar und generifiziert. Nach Generifizierung bleiben einer Klasse überwiegend nur noch konfigurierbare Anteile vorbehalten. Eine Konfiguration wurde als Refactor geplant und soll von dort aus zu einer zentralen Konfiguration hochgereicht werden.\newline

Im Projekt behandelte Metriken wurden aufgefasst mit einer Klasse Metrik, Metrik-Thresholds, Metrik-Registry, Metrik-Registry-Defintions deren Name die Funktion hinreichend vorgibt.\newline

Integriert wurden Lösungen von Herr M.Sc. Mavani der sich zur BNN Auswertung für ein Python Pyro Jax Backend entschieden und damit Lösungen nach den Pyroworkflows für Hamilton Monte Carlo (HMC), Stochastic Variational Inference (SVI) geliefert hat.


% \subsection*{Experiment}

\begin{table}[htbp]
\centering
\begin{tabularx}{\textwidth}{|l|l|l|l|l|l|l|X|}
\hline
 & \texttt{nll} & \texttt{abs} & \texttt{mse} & \texttt{kl} & \texttt{scaled} & \texttt{variational} & \texttt{full} \\
\hline
EDNN Accuracy & & & & & & & \\
\hline
BNN Accuracy & & & & & & & \\
\hline
\end{tabularx}
\caption{Zusammenfassung der Evidential Loss Varianten}
\end{table}


\section*{Visueller Abgleich \gls{Bayesianische neuronale Netze}-\gls{Evidenzbasierte neuronale Netze}}

\begin{figure}[!h]
  \centering
  \begin{minipage}{0.5\textwidth}
      \centering
      \includegraphics[width=\textwidth]{../figures/bnn_posterior_predictive.png}
      \caption{Ninand Mavani Visual Benchmark BNN HMC}
  \end{minipage}%
  \begin{minipage}{0.5\textwidth}
      \centering
      \includegraphics[width=\textwidth]{../figures/bnn_svi_predictive.png}
      \caption{Ninand Mavani Visual Benchmark BNN SVI}
  \end{minipage}
\end{figure}


% \subsection*{Schnittstelle}

% \subsection*{Benchmark EDNN}

% \subsection*{Benchmark BNN-EDNN}

% \begin{table}[h!]
% \centering
% \resizebox{\textwidth}{!}{
% \begin{tabular}{|c|c|c|c|c|}
% \hline
% \textbf{Sigma (Standard Deviation)} & \textbf{BNN Variance} & \textbf{BNN Accuracy (\%)} & \textbf{EDNN Variance} & \textbf{EDNN Accuracy (\%)} \\
% \hline
% 0 & 0.0 & 0 & 0.0 & 0 \\
% \hline
% \end{tabular}
% }
% \caption{Vergleich von BNN und EDNN mit Meta-Metrik im 2-Sigma-Bereich}
% \label{tab:BNN_EDNN_Comparison}
% \end{table}


\section{}

Abschließend sollen nach R11 die wichtigsten Ansätze und Ergebnisse der Arbeit in einem kurzen Jupyter Notebook zusammenfasst werden. 


\end{otherlanguage}
