% !TeX root = ../main.tex

\chapter{Diskussion}\label{chapter:diskussion}

Is Epistemic Uncertainty Faithfully Represented by Evidential Deep Learning Methods? ~\parencite{Jurgens.}.

Can you trust your ML metric? ~\parencite{Herd04082024}





% \section{Bewertung der Vertrauenswürdigkeit (Meta-Unsicherheit)}

% Der Begriff „Meta-Unsicherheit“ ist in der wissenschaftlichen Literatur nicht standardisiert etabliert. Ein Ziel der vorliegenden Arbeit besteht darin, über die klassische Trennung von AC und EC hinaus ein übergeordnetes Maß für die Vertrauenswürdigkeit modellbasierter Vorhersagen zu entwickeln – eine sogenannte Meta-Unsicherheit. Darunter wird die systematische Aggregation, Interpretation und Bewertung der vom Modell selbst generierten Unsicherheiten verstanden, mit dem Ziel, ein globales, kontextsensitives Vertrauensmaß abzuleiten. 

% Die Meta-Unsicherheit ergibt sich nicht als direkter Modelloutput, sondern als Funktion aus mehreren Unsicherheitskomponenten, die entlang einer Trajektorie, eines Szenarios oder im Rahmen eines gesamten Validierungslaufs kumulativ betrachtet werden. Typische Einflussgrößen sind dabei:
% \begin{itemize}
%   \item die lokale Evidenz (\(\alpha^{-1}\)) des Modells in Kombination mit der Vorhersagegüte (\(\left| y - \mu \right|\)),
%   \item die Verteilung epistemischer Varianz über den Eingaberaum,
%   \item strukturelle Merkmale wie Häufung unsicherer Bereiche oder abrupte Evidenzsprünge (z.\,B. entlang eines Pfades),
%   \item sowie Metriken wie die Jensen-Shannon-Divergenz zwischen aufeinanderfolgenden Unsicherheitsverteilungen oder Modellen.
% \end{itemize}

% Zur Erfassung der Meta-Unsicherheit kann ein aggregiertes Risikomaß definiert werden, das z.\,B. eine gewichtete Summe lokaler Unsicherheiten, Varianzgradienten oder Nichtkalibrierbarkeit darstellt. In der Praxis liefert ein solches Maß eine Einschätzung darüber, wie konsistent, robust und erklärbar die Vorhersagen eines Modells im Kontext der gestellten Aufgabe sind.

% Gerade in sicherheitskritischen Szenarien – etwa in der Validierung autonomer Systeme oder der simulationsbasierten Entscheidungsunterstützung – liefert Meta-Unsicherheit einen entscheidenden Beitrag zur Vertrauensbewertung. Sie ergänzt klassische Metriken der Modellgüte durch eine interpretierbare, quantitative Aussage über die Zuverlässigkeit des Modells selbst – nicht nur dessen Ausgabe.