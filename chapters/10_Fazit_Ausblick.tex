% !TeX root = ../main.tex

\chapter{Fazit}\label{chapter:fazit}



\begin{otherlanguage}{american}
%
%
%
\end{otherlanguage}



\begin{otherlanguage}{ngerman}
In diesem Abschnitt sollen zentrale Erkenntnisse der Arbeit zusammengefasst werden und in Perspektive gestellt werden für Weiterentwicklungen, Anwendungsfelder.

\section{Zusammenfassung zentraler Erkenntnisse}

Im Rahmen dieser Arbeit wurde ein evidenzbasiertes Verfahren zur Unsicherheitsquantifizierung in \gls{machinelearning}-basierten Surrogatmodellen methodisch angewandt und exemplarisch umgesetzt. Durch den Einsatz eines \gls{Evidenzbasierte neuronale Netze} konnten sowohl \gls{Aleatorische Unsicherheit} als auch \gls{Epistemische Unsicherheit} direkt aus den Eingabedaten gelernt und getrennt interpretiert werden. Die modellierten Unsicherheiten erwiesen sich als differenzierbar, kalibrierbar und visuell interpretierbar im Kontext von Szenarien, die mit einer Kritikalitätsfunktion belegt sind.

Herr Autenrieths Kritikalitätsräume als zu untersuchender Eingaberaum sind ein statisches System auf Basis algebraischer Gleichungen zum Simulieren von Nichtlinearitäten belegt mit pseudo-dynamischer Störfunktion. 

\[
y(t) = f(u(t)) + \epsilon(t)
\]

Der Kritikalitätsraum ist statisch, da der Ausgang ausschließlich vom aktuellen Eingang abhängt. Es enthält in der Thesis weder Zustandsgrößen noch Differenzialgleichungen, weist keinen Zeitverzug auf und besitzt kein Speicherverhalten. Die additive Störfunktion $\epsilon(t)$ kann als pseudo-dynamisch interpretiert werden, da sie zeitveränderlich ist, jedoch selbst kein eigenes Systemverhalten mit innerer Dynamik modelliert.

Obwohl ein \gls{Evidenzbasierte neuronale Netze} formal ein statisches Feedforward-System ist, kann es zur Laufzeit die Effekte dynamischer Systeme approximieren, etwa durch seine nichtlineare Projektion in die Unsicherheitsparameter. Die während des Trainings gelernten Gewichte speichern dabei ein implizites Gedächtnis über die Struktur der Eingaben. Dieses speicherartige Verhalten betrifft jedoch nicht den Inferenzprozess selbst, der zustandslos erfolgt.

Das formal statische Verhalten beider Systeme spricht - ohne formalen inneren Zustand sowie Dynamik des Systems - gegen die klassische Belegung von Steuer-, Beobacht- und damit Regelbarkeit der Systeme aus der Regelungstechnik.

In der Arbeit wurden in Bezug auf die Beobachtbar- sowie Steuerbarkeit die folgende Fragen positiv beantwortet: Erstens, kann das ENN durch geeignete Parametrisierung so gesteuert werden, dass es die Struktur des Kritikalitätsraums (zum Beispiel Unsicherheitsniveaus oder Schwellen) korrekt annähert? Damit ist es funktional steuerbar im Parameterraum. Zweitens, kann ich aus den Ausgaben des \gls{Evidenzbasierte neuronale Netze} (zum Beispiel Varianzen, Evidenz, Kritikalität) auf Strukturen im zugrunde liegenden Kritikalitätsraum schließen? Somit ist es eine beobachtbare Approximation, obwohl das System an sich zustandslos ist.

Die Kombination von Unsicherheitsmetriken, Evidenzparametern und Maßzahlen ermöglicht aus dieser Thesis ermöglicht zudem die Entwicklung eines Meta-Unsicherheitsmaßes zur Bewertung der Vertrauenswürdigkeit von Szenarien. Der Vergleich mit \gls{Bayesianische neuronale Netze}-Modellen zeigte, dass evidenzbasierte Ansätze nicht nur effizienter, sondern auch deutlich robuster gegenüber struktureller Unsicherheit agieren können. Insgesamt demonstriert die Arbeit das Potenzial evidenzieller Verfahren zur methodisch fundierten und anwendungsnahen Bewertung datengetriebener Modelle unter Unsicherheit.



\section{Ausblick}

\paragraph{EDL-Meta-Modelle, Realzeit-Validierung, Transfer auf andere Systeme} Aufbauend auf den in dieser Arbeit entwickelten Methoden zur evidenzbasierten Unsicherheitsquantifizierung ergeben sich mehrere weiterführende Forschungsperspektiven. Ein naheliegender Schritt ist die Entwicklung sogenannter EDL-Meta-Modelle, die nicht nur punktweise Unsicherheiten schätzen, sondern übergeordnetes Verhalten aggregieren und daraus adaptive Steuerungs- oder Bewertungssignale ableiten. Solche Modelle könnten als unsicherheitsregelnde Komponente in größeren Systemarchitekturen fungieren und beispielsweise dynamisch zwischen Regelstrategien oder Szenarien umschalten.
Ein weiterer vielversprechender Forschungspfad liegt in der Integration der EDL-Methodik in Realzeit-Validierungsumgebungen, insbesondere in Verbindung mit MPC oder szenariobasierten Testframeworks. Hierbei müssten nicht nur Vorhersagewerte, sondern auch die resultierenden Unsicherheiten in Echtzeit prozessiert und rückgeführt werden – etwa zur dynamischen Anpassung von Sicherheitsmargen oder zur intelligenten Szenarienselektion.
Darüber hinaus eröffnet sich das Potenzial zur Übertragung der Methodik auf andere technische Domänen, in denen datengetriebene Modelle in sicherheitsrelevante Entscheidungsprozesse eingebunden sind. Dazu zählen unter anderem autonome Landfahrzeuge, Robotik, industrielle Prozessüberwachung oder die adaptive Steuerung cyber-physischer Systeme. Entscheidend für einen erfolgreichen Transfer ist dabei die kontextspezifische Anpassung der Merkmalsextraktion und die Auswahl geeigneter Evidenzstrukturen.
Insgesamt bietet die Kombination aus transparent modellierter Unsicherheit, mathematisch fundierter Aggregation und systemischer Interpretierbarkeit eine zukunftsweisende Grundlage für robuste, erklärbare und vertrauenswürdige KI-basierte Systeme auf dem diese Arbeit aufsetzen konnte, um es beispielhaft anzuwenden.




\end{otherlanguage}