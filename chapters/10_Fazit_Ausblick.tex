% !TeX root = ../main.tex

\chapter{Fazit}\label{chapter:fazit}



\begin{otherlanguage}{american}
%
%
%
\end{otherlanguage}



\begin{otherlanguage}{ngerman}
In diesem Abschnitt sollen zentrale Erkenntnisse der Arbeit zusammengefasst werden und in Perspektive gestellt werden für Weiterentwicklungen, Anwendungsfelder.

\section{Zusammenfassung zentraler Erkenntnisse}

Im Rahmen dieser Arbeit wurde ein evidenzbasiertes Verfahren zur Unsicherheitsquantifizierung in machine-learning-basierten Surrogatmodellen methodisch angewandt und exemplarisch umgesetzt. Durch den Einsatz eines EDN konnten sowohl AC als auch EC direkt aus den Eingabedaten gelernt und getrennt interpretiert werden. Die modellierten Unsicherheiten erwiesen sich als differenzierbar, kalibrierbar und visuell interpretierbar – insbesondere im Kontext sicherheitskritischer Szenarien %[und ergaben einen Ausblick auf Steuerbarkeit].
Die Kombination von Unsicherheitsmetriken, Evidenzparametern und aggregierten Maßzahlen ermöglichte zudem die Entwicklung eines Meta-Unsicherheitsmaßes zur Bewertung der Vertrauenswürdigkeit ganzer Szenarien. %Der Vergleich mit klassischen BNN-Modellen zeigte, dass evidenzbasierte Ansätze nicht nur effizienter, sondern auch deutlich robuster gegenüber struktureller Unsicherheit agieren können.
% Insgesamt demonstriert die Arbeit das Potenzial evidenzieller Verfahren zur methodisch fundierten und anwendungsnahen Bewertung datengetriebener Modelle unter Unsicherheit.

\section{Ausblick}

% EDL-Meta-Modelle, Realzeit-Validierung, Transfer auf andere Systeme

% Aufbauend auf den in dieser Arbeit entwickelten Methoden zur evidenzbasierten Unsicherheitsquantifizierung ergeben sich mehrere weiterführende Forschungsperspektiven. Ein naheliegender Schritt ist die Entwicklung sogenannter EDL-Meta-Modelle, die nicht nur punktweise Unsicherheiten schätzen, sondern übergeordnetes Verhalten aggregieren und daraus adaptive Steuerungs- oder Bewertungssignale ableiten. Solche Modelle könnten als unsicherheitsregelnde Komponente in größeren Systemarchitekturen fungieren und beispielsweise dynamisch zwischen Regelstrategien oder Szenarien umschalten.
% Ein weiterer vielversprechender Forschungspfad liegt in der Integration der EDL-Methodik in Realzeit-Validierungsumgebungen, insbesondere in Verbindung mit MPC oder szenariobasierten Testframeworks. Hierbei müssten nicht nur Vorhersagewerte, sondern auch die resultierenden Unsicherheiten in Echtzeit prozessiert und rückgeführt werden – etwa zur dynamischen Anpassung von Sicherheitsmargen oder zur intelligenten Szenarienselektion.
% Darüber hinaus eröffnet sich das Potenzial zur Übertragung der Methodik auf andere technische Domänen, in denen datengetriebene Modelle in sicherheitsrelevante Entscheidungsprozesse eingebunden sind. Dazu zählen unter anderem autonome Landfahrzeuge, Robotik, industrielle Prozessüberwachung oder die adaptive Steuerung cyber-physischer Systeme. Entscheidend für einen erfolgreichen Transfer ist dabei die kontextspezifische Anpassung der Merkmalsextraktion und die Auswahl geeigneter Evidenzstrukturen.
% Insgesamt bietet die Kombination aus transparent modellierter Unsicherheit, mathematisch fundierter Aggregation und systemischer Interpretierbarkeit eine zukunftsweisende Grundlage für robuste, erklärbare und vertrauenswürdige KI-basierte Systeme auf dem diese Arbeit aufsetzen konnte, um es beispielhaft anzuwenden.

\end{otherlanguage}