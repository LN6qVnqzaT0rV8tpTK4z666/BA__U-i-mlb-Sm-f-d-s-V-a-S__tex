%%%%%%%%%%%%%%%%%%%%%%%%%%%%%%%%%%%%%%%%%%%%%%%%%%%%%%%%%%%%%%%%%%%%%%%%%%%%%%%%%%%%%%%%%%%%%%%%%%%%%%
% IAT Template Document
%%%%%%%%%%%%%%%%%%%%%%%%%%%%%%%%%%%%%%%%%%%%%%%%%%%%%%%%%%%%%%%%%%%%%%%%%%%%%%%%%%%%%%%%%%%%%%%%%%%%%%

\NeedsTeXFormat{LaTeX2e}
% \ProvidesClass{iat}[2020/05/15 v1 Template]
% \PassOptionsToClass{%
% 	a4paper,%
% 	11pt,%
% 	parskip=half,%
% 	%twoside,%
% 	%openright,%
% 	BCOR=3mm,%
% }{scrreprt}
% \ProcessOptions\relax
% \LoadClass{scrreprt}

%%%%%%%%%%%%%%%%%%%%%%%%%%%%%%%%%%%%%%%%%%%%%%%%%%%%%%%%%%%%%%%%%%%%%%%%%%%%%%%%%%%%%%%%%%%%%%%%%%%%%%
% TUM Template
%%%%%%%%%%%%%%%%%%%%%%%%%%%%%%%%%%%%%%%%%%%%%%%%%%%%%%%%%%%%%%%%%%%%%%%%%%%%%%%%%%%%%%%%%%%%%%%%%%%%%%

\PassOptionsToPackage{table,svgnames,dvipsnames}{xcolor}

%%%%%%%%%%%%%%%%%%%%%%%%%%%%%%%%%%%%%%%%%%%%%%%%%%%%%%%%%%%%%%%%%%%%%%%%%%%%%%%%%%%%%%%%%%%%%%%%%%%%%%
% TUM Packages
%%%%%%%%%%%%%%%%%%%%%%%%%%%%%%%%%%%%%%%%%%%%%%%%%%%%%%%%%%%%%%%%%%%%%%%%%%%%%%%%%%%%%%%%%%%%%%%%%%%%%%

\usepackage[utf8]{inputenc}
\usepackage[T1]{fontenc}
% \usepackage[sc]{mathpazo}
\usepackage[ngerman]{babel}
\usepackage{booktabs}
\usepackage{amsmath}
\usepackage{amsfonts}
\usepackage{amssymb}
\usepackage{amstext}
\usepackage[autostyle]{csquotes}
\usepackage[%
  backend=biber,
  bibencoding=UTF8,
  urldate=comp,
  url=false,
  style=ieee, % alphabetic
  maxnames=4,
  minnames=3,
  maxbibnames=99,
  giveninits,
  uniquename=init,
  %keywords=true
]{biblatex} % TODO: adapt citation style
\usepackage[acronym]{glossaries} % Glossar
\usepackage{graphicx}
\usepackage{geometry}
\usepackage{scrhack} % necessary for listings package
\usepackage{listings}
\usepackage{lstautogobble}
\usepackage{tikz}
\usepackage{pgfplots}
\usepackage{pgfplotstable}
\usepackage[final]{microtype}
\usepackage[%
justification=raggedright,
singlelinecheck=false,
font=footnotesize,
format=hang,
indent=1cm
]{caption}
\usepackage[hidelinks]{hyperref}
\usepackage{comment}
\usepackage{longtable} % Wenn Tabelle zu lang für Seite ist
%\usepackage{array}     % Tabellensteuerung
\usepackage{tabularx}  % Tabellensteuerung
\usepackage{enumitem}
\usepackage{nag}
\usepackage{silence}
\WarningFilter{latex}{Command \showhyphens has changed}
\WarningFilter{datatool-base}{No `datatool' support for dialect `ngerman'}

% \usepackage{lmodern}
% \usepackage{siunitx}
% \AtBeginDocument{\RenewCommandCopy\qty\SI}
% \usepackage{physics}
% \usepackage{scrlayer-scrpage}
% \usepackage[version=4]{mhchem}
% \usepackage{mathtools}
% \usepackage{float}
\usepackage{xcolor}
\usepackage{mdframed}
% \usepackage{trfsigns}
% \usepackage{capt-of}

%%%%%%%%%%%%%%%%%%%%%%%%%%%%%%%%%%%%%%%%%%%%%%%%%%%%%%%%%%%%%%%%%%%%%%%%%%%%%%%%%%%%%%%%%%%%%%%%%%%%%%
% IAT Packages
%%%%%%%%%%%%%%%%%%%%%%%%%%%%%%%%%%%%%%%%%%%%%%%%%%%%%%%%%%%%%%%%%%%%%%%%%%%%%%%%%%%%%%%%%%%%%%%%%%%%%%

\usepackage{pdfpages} %The following package is needed to implement pdfs for conformity decl.  

%%%%%%%%%%%%%%%%%%%%%%%%%%%%%%%%%%%%%%%%%%%%%%%%%%%%%%%%%%%%%%%%%%%%%%%%%%%%%%%%%%%%%%%%%%%%%%%%%%%%%%
% IAT Template Packages (cleared of doubles)
%%%%%%%%%%%%%%%%%%%%%%%%%%%%%%%%%%%%%%%%%%%%%%%%%%%%%%%%%%%%%%%%%%%%%%%%%%%%%%%%%%%%%%%%%%%%%%%%%%%%%%

% \PassOptionsToPackage{right=5cm,top=3.5cm,bottom=4cm,head=25pt,foot=21pt, footskip=1.1cm}{geometry}
% \PassOptionsToPackage{hidelinks,breaklinks}{hyperref}	% add "linktocpage" option for page number instead of chapter name
% \PassOptionsToPackage{libertine}{newtxmath}
% \PassOptionsToPackage{headsepline=3pt, footsepline, automark, autooneside=false}{scrlayer-scrpage}
% \PassOptionsToPackage{english,main=ngerman}{babel}
% \PassOptionsToPackage{detect-all}{siunitx}
% \PassOptionsToPackage{style=ieee, urldate=comp, bibencoding=UTF8, backend=biber}{biblatex}
% \PassOptionsToPackage{T1}{fontenc}
% \PassOptionsToPackage{utf8}{inputenc}

\RequirePackage{babel}
% \RequirePackage{csquotes}
% \RequirePackage{microtype}
\RequirePackage{setspace}% für Onehalfspacing, ggf irgendwie erstetzen
\RequirePackage{libertine}
\RequirePackage{newtxmath}
\RequirePackage{color}
% \RequirePackage{textcomp}
% \RequirePackage{glossaries}
\RequirePackage{scrlayer-scrpage}

%%%%%%%%%%%%%%%%%%%%%%%%%%%%%%%%%%%%%%%%%%%%%%%%%%%%%%%%%%%%%%%%%%%%%%%%%%%%%%%%%%%%%%%%%%%%%%%%%%%%%%
% TUM Setup Packages
%%%%%%%%%%%%%%%%%%%%%%%%%%%%%%%%%%%%%%%%%%%%%%%%%%%%%%%%%%%%%%%%%%%%%%%%%%%%%%%%%%%%%%%%%%%%%%%%%%%%%%

% \sisetup{locale=DE}
% \sisetup{per-mode = symbol-or-fraction}
% \sisetup{separate-uncertainty=true}
% \DeclareSIUnit\year{a}
% \DeclareSIUnit\clight{c}
\mdfdefinestyle{exercise}{
	backgroundcolor=black!10,roundcorner=8pt,hidealllines=true,nobreak
}

\setlength{\parindent}{0pt}

% \makeatletter
% \g@addto@macro\@makechapterhead{\@afterindentfalse}
% \makeatother

\makeglossaries

\bibliography{bibliography}
\addbibresource{bibliography.bib}

\setkomafont{disposition}{\normalfont\bfseries} % use serif font for headings
\linespread{1.05} % adjust line spread for mathpazo font

% Add table of contents to PDF bookmarks
\BeforeTOCHead[toc]{{\cleardoublepage\pdfbookmark[0]{\contentsname}{toc}}}

%%%%%%%%%%%%%%%%%%%%%%%%%%%%%%%%%%%%%%%%%%%%%%%%%%%%%%%%%%%%%%%%%%%%%%%%%%%%%%%%%%%%%%%%%%%%%%%%%%%%%%
% Universität Bremen: CI Colors
%%%%%%%%%%%%%%%%%%%%%%%%%%%%%%%%%%%%%%%%%%%%%%%%%%%%%%%%%%%%%%%%%%%%%%%%%%%%%%%%%%%%%%%%%%%%%%%%%%%%%%

% Uni Bremen Corporate Design Colors (Core)
\definecolor{UniBremenRed}{HTML}{CC071E}
\definecolor{UniBremenDarkGray}{HTML}{333333}
\definecolor{UniBremenLightGray}{HTML}{DCDCDC}
\definecolor{UniBremenWhite}{HTML}{FFFFFF}
\definecolor{UniBremenBlack}{HTML}{000000}
\definecolor{UniBremenBlue}{HTML}{003DA5}

% Derived or custom Bremen-inspired colors
\definecolor{UniBremenBlueDark}{HTML}{002966} % darker Bremen blue
\definecolor{UniBremenBlueDarker}{HTML}{001A33} % even darker Bremen blue
\definecolor{UniBremenGray}{HTML}{808080} % mid gray, not official
\definecolor{UniBremenAccentGray}{HTML}{DAD7CB} % custom warm gray
\definecolor{UniBremenOrange}{HTML}{CC3200} % orange derived from Bremen red
\definecolor{UniBremenGreen}{HTML}{A2AD00} % custom green, not in CD
\definecolor{UniBremenLightBlue}{HTML}{98BFE0} % light tint of Bremen blue
\definecolor{UniBremenAccentBlue}{HTML}{4C88BA} % lighter Bremen blue

% Settings for pgfplots
\pgfplotsset{compat=newest}
\pgfplotsset{
	% For available color names, see http://www.latextemplates.com/svgnames-colors
	cycle list={
		UniBremenBlue\\
		UniBremenOrange\\
		UniBremenGreen\\
		UniBremenBlueDarker\\
		UniBremenDarkGray\\
	},
}

% Settings for lstlistings
\lstset{%
	basicstyle=\ttfamily,
	columns=fullflexible,
	autogobble,
	keywordstyle=\bfseries\color{UniBremenBlue},
	stringstyle=\color{UniBremenGreen}
}

\addto\captionsngerman{%
  \renewcommand{\contentsname}{Inhaltsverzeichnis}
  \renewcommand{\listfigurename}{Liste der Abbildungen}
  \renewcommand{\listtablename}{Liste der Tabellen}
  \renewcommand{\abstractname}{Kurzfassung}
  \renewcommand{\bibname}{Literaturverzeichnis}
  \renewcommand{\glossaryname}{Glossar}
  \renewcommand{\acronymname}{Abkürzungsverzeichnis}
  \renewcommand{\indexname}{Stichwortverzeichnis}
}

%%%%%%%%%%%%%%%%%%%%%%%%%%%%%%%%%%%%%%%%%%%%%%%%%%%%%%%%%%%%%%%%%%%%%%%%%%%%%%%%%%%%%%%%%%%%%%%%%%%%%%
% New Environments
%%%%%%%%%%%%%%%%%%%%%%%%%%%%%%%%%%%%%%%%%%%%%%%%%%%%%%%%%%%%%%%%%%%%%%%%%%%%%%%%%%%%%%%%%%%%%%%%%%%%%%

\newenvironment{formelsammlung}{
    \newpage
    \twocolumn
    \section*{Formelsammlung}
}{}

\newenvironment{noindentquote}
{\list{}{\leftmargin=0pt\rightmargin=0pt}\item[]}
{\endlist}

\newenvironment{pseudoitemize}
  {\begin{list}{}{\leftmargin=2em \itemindent=0pt \itemsep=0pt \topsep=0pt}}
  {\end{list}}

%%%%%%%%%%%%%%%%%%%%%%%%%%%%%%%%%%%%%%%%%%%%%%%%%%%%%%%%%%%%%%%%%%%%%%%%%%%%%%%%%%%%%%%%%%%%%%%%%%%%%%
% Glossar
%%%%%%%%%%%%%%%%%%%%%%%%%%%%%%%%%%%%%%%%%%%%%%%%%%%%%%%%%%%%%%%%%%%%%%%%%%%%%%%%%%%%%%%%%%%%%%%%%%%%%%

\newglossaryentry{accuracy}{
  name={Accuracy},
  description={Maß für die Treffergenauigkeit eines Modells, definiert als Anteil korrekt klassifizierter Vorhersagen an allen Vorhersagen}
}

\newglossaryentry{Aleatorische Unsicherheit}{
  name=AC,
  description={Datenunsicherheit}
}

\newglossaryentry{Autonomous Underwater Vehicle}{
  name=AUV,
  description={Autonomes Unterwasser Fahrzeug}
}

\newglossaryentry{bayesianischeinferenz}{
  name={BI},
  description={Schätzung unbekannter Größen durch Kombination von Daten und Vorwissen mittels Bayes-Theorem}
}

\newglossaryentry{Bundesministerium für Bildung und Forschung}{
  name=BMBF,
  description={Bundesministerium für Bildung und Forschung}
}

\newglossaryentry{Bayesianische neuronale Netze}{
  name=BNN,
  description={Bayesianische neuronale Netze, die Unsicherheit in den Modellparametern durch probabilistische Inferenz berücksichtigen, um eine Verteilung der möglichen Modellparameter statt eines festen Satzes von Parametern zu lernen}
}

\newglossaryentry{Conformal Prediction}{
  name=CP,
  description={Ein Verfahren zur Unsicherheitsquantifizierung, das es ermöglicht, Vorhersagen mit einer kontrollierten Fehlerwahrscheinlichkeit zu treffen. Conformal Prediction erzeugt für jede Vorhersage ein Konfidenzintervall, das die Unsicherheit in den Modellvorhersagen darstellt. Es basiert auf der Idee, dass die Vorhersage als "konform" zu den Trainingsdaten betrachtet wird, wenn sie eine bestimmte Konsistenzbedingung erfüllt. Dieses Verfahren ist besonders nützlich in Situationen, in denen die Datenverteilung unbekannt oder nicht unabhängig identisch verteilt ist}
}

\newglossaryentry{Deutschen Zentrum für Luft- und Raumfahrt}{
  name=DLR,
  description={Deutsches Zentrum für Luft- und Raumfahrt}
}

\newglossaryentry{Deutschen Allianz für Meeresforschung}{
  name=DAM,
  description={Deutschen Allianz für Meeresforschung}
}

\newglossaryentry{Epistemische Unsicherheit}{
  name=EC,
  description={Modellunsicherheit}
}

\newglossaryentry{Evidenzbasierte neuronale Netze}{
  name=ENN,
  description={Evidenzbasierte neuronale Netze, die Unsicherheit nicht nur in den Modellparametern, sondern auch in den Daten selbst durch die Verwendung einer evidenzbasierten Inferenztechnik berücksichtigen. Statt eine feste Wahrscheinlichkeitsverteilung der Modellparameter zu lernen, modellieren ENNs die Unsicherheit als Evidenz, die die Unsicherheit sowohl der Daten als auch der Modellparameter direkt widerspiegelt. Diese Methode ermöglicht eine robustere Quantifizierung von Unsicherheit und verbessert die Handhabung von Unsicherheit in realen Anwendungsszenarien}
}

\newglossaryentry{Gaußsche Prozessregression}{
  name=GPR,
  description={Nichtparametrisches, probabilistisches Modell zur Vorhersage,  Quantifizierung von Unsicherheiten in Regressionen (Vorhersagen als Verteilungen statt als Punktwerte)}
}

\newglossaryentry{machinelearning}{
  name={ML},
  description={Teilgebiet der Künstlichen Intelligenz, das Systeme befähigt, Muster in Daten zu erkennen und daraus zu lernen, ohne explizit programmiert zu sein}
}

\newglossaryentry{Out-of-Distribution}{
  name=OOD,
  description={Out-Of-Distribution-Data bezeichnet Datenpunkte, die aus einer anderen Verteilung stammen als derjenigen, auf der das Modell trainiert wurde, und deren Erkennung für die zuverlässige Unsicherheitsquantifizierung essenziell ist}
}

\newglossaryentry{surrogat}{
  name=Surrogatmodell,
  description={Ein Ersatzmodell zur Approximation komplexer Systeme}
}

\newglossaryentry{relativeuncertaintyindex}{
  name={RUI},
  description={Maß für die Unsicherheitsbewertung im Verhältnis zur mittleren Vorhersage oder zu Referenzwerten}
}

\newglossaryentry{variationalinference}{
  name={VI},
  description={Approximation komplexer Wahrscheinlichkeitsverteilungen durch Optimierung einfacherer Verteilunge (Minimierung Kullback-Leibler)}
}
