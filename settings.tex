%%%%%%%%%%%%%%%%%%%%%%%%%%%%%%%%%%%%%%%%%%%%%%%%%%%%%%%%%%%%%%%%%%%%%%%%%%%%%%%%%%%%%%%%%%%%%%%%%%%%%%
% IAT Template Document
%%%%%%%%%%%%%%%%%%%%%%%%%%%%%%%%%%%%%%%%%%%%%%%%%%%%%%%%%%%%%%%%%%%%%%%%%%%%%%%%%%%%%%%%%%%%%%%%%%%%%%

\NeedsTeXFormat{LaTeX2e}
% \ProvidesClass{iat}[2020/05/15 v1 Template]
% \PassOptionsToClass{%
% 	a4paper,%
% 	11pt,%
% 	parskip=half,%
% 	%twoside,%
% 	%openright,%
% 	BCOR=3mm,%
% }{scrreprt}
% \ProcessOptions\relax
% \LoadClass{scrreprt}

%%%%%%%%%%%%%%%%%%%%%%%%%%%%%%%%%%%%%%%%%%%%%%%%%%%%%%%%%%%%%%%%%%%%%%%%%%%%%%%%%%%%%%%%%%%%%%%%%%%%%%
% TUM Template
%%%%%%%%%%%%%%%%%%%%%%%%%%%%%%%%%%%%%%%%%%%%%%%%%%%%%%%%%%%%%%%%%%%%%%%%%%%%%%%%%%%%%%%%%%%%%%%%%%%%%%

\PassOptionsToPackage{table,svgnames,dvipsnames}{xcolor}

%%%%%%%%%%%%%%%%%%%%%%%%%%%%%%%%%%%%%%%%%%%%%%%%%%%%%%%%%%%%%%%%%%%%%%%%%%%%%%%%%%%%%%%%%%%%%%%%%%%%%%
% TUM Packages
%%%%%%%%%%%%%%%%%%%%%%%%%%%%%%%%%%%%%%%%%%%%%%%%%%%%%%%%%%%%%%%%%%%%%%%%%%%%%%%%%%%%%%%%%%%%%%%%%%%%%%

\usepackage[utf8]{inputenc}
\usepackage[T1]{fontenc}
% \usepackage[sc]{mathpazo}
\usepackage[ngerman]{babel}
\usepackage{booktabs}
\usepackage{amsmath}
\usepackage{amsfonts}
\usepackage{amssymb}
\usepackage{amstext}
\usepackage[autostyle]{csquotes}
\usepackage[%
  backend=biber,
  bibencoding=UTF8,
  urldate=comp,
  url=false,
  style=ieee, % alphabetic
  maxnames=4,
  minnames=3,
  maxbibnames=99,
  giveninits,
  uniquename=init,
  %keywords=true
]{biblatex} % TODO: adapt citation style
\usepackage[acronym]{glossaries} % Glossar
\usepackage{graphicx}
\usepackage{geometry}
\usepackage{scrhack} % necessary for listings package
\usepackage{listings}
\usepackage{lstautogobble}
\usepackage{tikz}
\usepackage{pgfplots}
\usepackage{pgfplotstable}
\usepackage[final]{microtype}
\usepackage[%
justification=raggedright,
singlelinecheck=false,
font=footnotesize,
format=hang,
indent=1cm
]{caption}
\usepackage[hidelinks]{hyperref}
\usepackage{comment}
\usepackage{longtable} % Wenn Tabelle zu lang für Seite ist
%\usepackage{array}     % Tabellen
\usepackage{tabularx}  % Tabellen
\usepackage{enumitem}
\usepackage{nag}
\usepackage{silence}
\WarningFilter{latex}{Command \showhyphens has changed}
\WarningFilter{datatool-base}{No `datatool' support for dialect `ngerman'}
\usepackage{tabularray} % Tabellen 
\usepackage{multirow} % Tabellen
\usepackage{longtable} % Tabellen
\usepackage{ltablex} % Tabellen
\usepackage{makecell} % Tabellen
\keepXColumns % verhindert, dass ltablex die Spaltenanzahl ändert

% \usepackage{lmodern}
% \usepackage{siunitx}
% \AtBeginDocument{\RenewCommandCopy\qty\SI}
% \usepackage{physics}
% \usepackage{scrlayer-scrpage}
% \usepackage[version=4]{mhchem}
% \usepackage{mathtools}
% \usepackage{float}
\usepackage{xcolor}
\usepackage{mdframed}
% \usepackage{trfsigns}
% \usepackage{capt-of}

%%%%%%%%%%%%%%%%%%%%%%%%%%%%%%%%%%%%%%%%%%%%%%%%%%%%%%%%%%%%%%%%%%%%%%%%%%%%%%%%%%%%%%%%%%%%%%%%%%%%%%
% IAT Packages
%%%%%%%%%%%%%%%%%%%%%%%%%%%%%%%%%%%%%%%%%%%%%%%%%%%%%%%%%%%%%%%%%%%%%%%%%%%%%%%%%%%%%%%%%%%%%%%%%%%%%%

\usepackage{pdfpages} %The following package is needed to implement pdfs for conformity decl.  

%%%%%%%%%%%%%%%%%%%%%%%%%%%%%%%%%%%%%%%%%%%%%%%%%%%%%%%%%%%%%%%%%%%%%%%%%%%%%%%%%%%%%%%%%%%%%%%%%%%%%%
% IAT Template Packages (cleared of doubles)
%%%%%%%%%%%%%%%%%%%%%%%%%%%%%%%%%%%%%%%%%%%%%%%%%%%%%%%%%%%%%%%%%%%%%%%%%%%%%%%%%%%%%%%%%%%%%%%%%%%%%%

% \PassOptionsToPackage{right=5cm,top=3.5cm,bottom=4cm,head=25pt,foot=21pt, footskip=1.1cm}{geometry}
% \PassOptionsToPackage{hidelinks,breaklinks}{hyperref}	% add "linktocpage" option for page number instead of chapter name
% \PassOptionsToPackage{libertine}{newtxmath}
% \PassOptionsToPackage{headsepline=3pt, footsepline, automark, autooneside=false}{scrlayer-scrpage}
% \PassOptionsToPackage{english,main=ngerman}{babel}
% \PassOptionsToPackage{detect-all}{siunitx}
% \PassOptionsToPackage{style=ieee, urldate=comp, bibencoding=UTF8, backend=biber}{biblatex}
% \PassOptionsToPackage{T1}{fontenc}
% \PassOptionsToPackage{utf8}{inputenc}

\RequirePackage{babel}
% \RequirePackage{csquotes}
% \RequirePackage{microtype}
\RequirePackage{setspace}% für Onehalfspacing, ggf irgendwie erstetzen
\RequirePackage{libertine}
\RequirePackage{newtxmath}
\RequirePackage{color}
% \RequirePackage{textcomp}
% \RequirePackage{glossaries}
\RequirePackage{scrlayer-scrpage}

%%%%%%%%%%%%%%%%%%%%%%%%%%%%%%%%%%%%%%%%%%%%%%%%%%%%%%%%%%%%%%%%%%%%%%%%%%%%%%%%%%%%%%%%%%%%%%%%%%%%%%
% TUM Setup Packages
%%%%%%%%%%%%%%%%%%%%%%%%%%%%%%%%%%%%%%%%%%%%%%%%%%%%%%%%%%%%%%%%%%%%%%%%%%%%%%%%%%%%%%%%%%%%%%%%%%%%%%

% \sisetup{locale=DE}
% \sisetup{per-mode = symbol-or-fraction}
% \sisetup{separate-uncertainty=true}
% \DeclareSIUnit\year{a}
% \DeclareSIUnit\clight{c}
\mdfdefinestyle{exercise}{
	backgroundcolor=black!10,roundcorner=8pt,hidealllines=true,nobreak
}

\setlength{\parindent}{0pt}

% \makeatletter
% \g@addto@macro\@makechapterhead{\@afterindentfalse}
% \makeatother

\makeglossaries

\bibliography{bibliography}
\addbibresource{bibliography.bib}

\setkomafont{disposition}{\normalfont\bfseries} % use serif font for headings
\linespread{1.05} % adjust line spread for mathpazo font

% Add table of contents to PDF bookmarks
\BeforeTOCHead[toc]{{\cleardoublepage\pdfbookmark[0]{\contentsname}{toc}}}

%%%%%%%%%%%%%%%%%%%%%%%%%%%%%%%%%%%%%%%%%%%%%%%%%%%%%%%%%%%%%%%%%%%%%%%%%%%%%%%%%%%%%%%%%%%%%%%%%%%%%%
% Universität Bremen: CI Colors
%%%%%%%%%%%%%%%%%%%%%%%%%%%%%%%%%%%%%%%%%%%%%%%%%%%%%%%%%%%%%%%%%%%%%%%%%%%%%%%%%%%%%%%%%%%%%%%%%%%%%%

% Uni Bremen Corporate Design Colors (Core)
\definecolor{UniBremenRed}{HTML}{CC071E}
\definecolor{UniBremenDarkGray}{HTML}{333333}
\definecolor{UniBremenLightGray}{HTML}{DCDCDC}
\definecolor{UniBremenWhite}{HTML}{FFFFFF}
\definecolor{UniBremenBlack}{HTML}{000000}
\definecolor{UniBremenBlue}{HTML}{003DA5}

% Derived or custom Bremen-inspired colors
\definecolor{UniBremenBlueDark}{HTML}{002966} % darker Bremen blue
\definecolor{UniBremenBlueDarker}{HTML}{001A33} % even darker Bremen blue
\definecolor{UniBremenGray}{HTML}{808080} % mid gray, not official
\definecolor{UniBremenAccentGray}{HTML}{DAD7CB} % custom warm gray
\definecolor{UniBremenOrange}{HTML}{CC3200} % orange derived from Bremen red
\definecolor{UniBremenGreen}{HTML}{A2AD00} % custom green, not in CD
\definecolor{UniBremenLightBlue}{HTML}{98BFE0} % light tint of Bremen blue
\definecolor{UniBremenAccentBlue}{HTML}{4C88BA} % lighter Bremen blue

% Settings for pgfplots
\pgfplotsset{compat=newest}
\pgfplotsset{
	% For available color names, see http://www.latextemplates.com/svgnames-colors
	cycle list={
		UniBremenBlue\\
		UniBremenOrange\\
		UniBremenGreen\\
		UniBremenBlueDarker\\
		UniBremenDarkGray\\
	},
}

% Settings for lstlistings
\lstset{%
	basicstyle=\ttfamily,
	columns=fullflexible,
	autogobble,
	keywordstyle=\bfseries\color{UniBremenBlue},
	stringstyle=\color{UniBremenGreen}
}

\addto\captionsngerman{%
  \renewcommand{\contentsname}{Inhaltsverzeichnis}
  \renewcommand{\listfigurename}{Liste der Abbildungen}
  \renewcommand{\listtablename}{Liste der Tabellen}
  \renewcommand{\abstractname}{Kurzfassung}
  \renewcommand{\bibname}{Literaturverzeichnis}
  \renewcommand{\glossaryname}{Glossar}
  \renewcommand{\acronymname}{Abkürzungsverzeichnis}
  \renewcommand{\indexname}{Stichwortverzeichnis}
}

%%%%%%%%%%%%%%%%%%%%%%%%%%%%%%%%%%%%%%%%%%%%%%%%%%%%%%%%%%%%%%%%%%%%%%%%%%%%%%%%%%%%%%%%%%%%%%%%%%%%%%
% New Environments
%%%%%%%%%%%%%%%%%%%%%%%%%%%%%%%%%%%%%%%%%%%%%%%%%%%%%%%%%%%%%%%%%%%%%%%%%%%%%%%%%%%%%%%%%%%%%%%%%%%%%%

\newenvironment{formelsammlung}{
    \newpage
    \twocolumn
    \section*{Formelsammlung}
}{}

\newenvironment{noindentquote}
{\list{}{\leftmargin=0pt\rightmargin=0pt}\item[]}
{\endlist}

\newenvironment{pseudoitemize}
  {\begin{list}{}{\leftmargin=2em \itemindent=0pt \itemsep=0pt \topsep=0pt}}
  {\end{list}}

%%%%%%%%%%%%%%%%%%%%%%%%%%%%%%%%%%%%%%%%%%%%%%%%%%%%%%%%%%%%%%%%%%%%%%%%%%%%%%%%%%%%%%%%%%%%%%%%%%%%%%
% Glossar
%%%%%%%%%%%%%%%%%%%%%%%%%%%%%%%%%%%%%%%%%%%%%%%%%%%%%%%%%%%%%%%%%%%%%%%%%%%%%%%%%%%%%%%%%%%%%%%%%%%%%%

\newglossaryentry{accuracy}{
  name={Accuracy},
  description={\textit{Accuracy} ist ein Maß für die Treffergenauigkeit eines Modells, definiert als Anteil korrekt klassifizierter Vorhersagen an allen Vorhersagen}
}

\newglossaryentry{Aleatorische Unsicherheit}{
  name={AUQ},
  description={\textit{Aleatorische Unsicherheit} ist äquivalent zur Datenunsicherheit}
}

\newglossaryentry{aucpr}{
    name={AUC-PR},
    description={\textit{Area Under the Precision-Recall Curve.} Maß für die Qualität eines binären Klassifikators, speziell bei unausgeglichenen Klassenverteilungen. Misst die Fläche unter der Kurve, die den Zusammenhang zwischen Precision (Positiver Vorhersagewert) und Recall (Sensitivität) über verschiedene Schwellenwerte darstellt. Höhere Werte deuten auf bessere Unterscheidung zwischen Klassen hin}
}

\newglossaryentry{Autonomous Underwater Vehicle}{
  name={AUV},
  description={\textit{Autonomes Unterwasser Fahrzeug}}
}

\newglossaryentry{bayesianischeinferenz}{
  name={BI},
  description={\textit{Bayesianische Inferenz} ist eine Schätzung unbekannter Größen durch Kombination von Daten und Vorwissen mittels Bayes-Theorem}
}

\newglossaryentry{Bundesministerium für Bildung und Forschung}{
  name={BMBF},
  description={\textit{Bundesministerium für Bildung und Forschung}}
}

\newglossaryentry{Bayesianische neuronale Netze}{
  name={BNN},
  description={\textit{Bayesianische neuronale Netze} sind NN, welche Unsicherheit in den Modellparametern durch probabilistische Inferenz berücksichtigen, um eine Verteilung der möglichen Modellparameter statt eines festen Satzes von Parametern zu lernen}
}

\newglossaryentry{calibrationerror}{
  name={ECE},
  description={\textit{Calibration Error} ist eine Metrik zur Messung der Übereinstimmung zwischen vorhergesagten Wahrscheinlichkeiten eines Modells und den tatsächlichen beobachteten Häufigkeiten}
}

\newglossaryentry{Conformal Prediction}{
  name={CP},
  description={\textit{Konforme Vorhersagen} sind ein Verfahren zur Unsicherheitsquantifizierung, das es ermöglicht, Vorhersagen mit einer kontrollierten Fehlerwahrscheinlichkeit zu treffen. Konforme Verhersagen erzeugen für jede Vorhersage ein Konfidenzintervall, welches die Unsicherheit in den Modellvorhersagen darstellt. Es basiert auf der Idee, dass die Vorhersage als "konform" zu den Trainingsdaten betrachtet wird, wenn sie eine bestimmte Konsistenzbedingung erfüllt. Dieses Verfahren ist besonders nützlich in Situationen, in denen die Datenverteilung unbekannt oder nicht unabhängig identisch verteilt ist}
}

\newglossaryentry{CRPS}{
    name=CRPS,
    description={\textit{Continuous Ranked Probability Score} Maß zur Bewertung probabilistischer Vorhersagen, das die Distanz zwischen kumulativen Verteilungsfunktionen von Vorhersage und Beobachtung misst}
}

\newglossaryentry{DeepNeuralNetwork}{
  name={DNN},
  description={\textit{Deep Neural Network}; ein tiefes neuronales Netzwerk mit mehreren Schichten zwischen Eingabe und Ausgabe, das komplexe nichtlineare Zusammenhänge modellieren kann.}
}

\newglossaryentry{Deutschen Zentrum für Luft- und Raumfahrt}{
  name={DLR},
  description={\textit{Deutsches Zentrum für Luft- und Raumfahrt}}
}

\newglossaryentry{Deutschen Allianz für Meeresforschung}{
  name={DAM},
  description={\textit{Deutschen Allianz für Meeresforschung}}
}

\newglossaryentry{Epistemische Unsicherheit}{
  name={EUQ},
  description={\textit{Epistemische Unsicherheit} ist äquivalent zur Modellunsicherheit}
}

\newglossaryentry{EvidentialDeepLearning}{
  name={EDL},
  description={\textit{Evidential Deep Learning} ist Ansatz im Deep Learning, bei dem NN nicht nur Vorhersagen liefern, sondern auch eine Maßzahl für die Evidenz bzw. Sicherheit dieser Vorhersagen bereitstellen. EDL modelliert Unsicherheiten direkt durch die Schätzung von Parametern von Wahrscheinlichkeitsverteilungen und liefert dadurch interpretierbare Unsicherheitsmaße}
}

\newglossaryentry{Evidenzbasierte neuronale Netze}{
  name={ENN},
  description={\textit{Evidenzbasierte neuronale Netze} sind NN, die Unsicherheit nicht nur in den Modellparametern, sondern auch in den Daten selbst durch die Verwendung einer evidenzbasierten Inferenztechnik berücksichtigen. Statt eine feste Wahrscheinlichkeitsverteilung der Modellparameter zu lernen, modellieren ENNs die Unsicherheit als Evidenz, die die Unsicherheit sowohl der Daten als auch der Modellparameter direkt widerspiegelt. Diese Methode ermöglicht eine robustere Quantifizierung von Unsicherheit und verbessert die Handhabung von Unsicherheit in realen Anwendungsszenarien}
}

\newglossaryentry{Gaußsche Prozessregression}{
  name={GPR},
  description={\textit{Gaußsche Prozessregression} ist nichtparametrisches, probabilistisches Modell zur Vorhersage,  Quantifizierung von Unsicherheiten in Regressionen (Vorhersagen als Verteilungen statt als Punktwerte)}
}

\newglossaryentry{GroundTruth}{
  name={GT},
  description={\textit{Ground Truth} bezeichnet in der Datenanalyse und im maschinellen Lernen die tatsächlichen, als korrekt angenommenen Referenzwerte oder Labels, anhand derer die Qualität von Vorhersagen oder Modellen bewertet wird. Die Ground Truth dient als Maßstab für die Genauigkeit und Validierung von Algorithmen und Datenauswertungen}
}

\newglossaryentry{hmc}{
  name={HMC},
  description={\textit{Hamiltonian Monte Carlo}; ein Markov-Chain-Monte-Carlo-Verfahren (MCMC), das physikalische Konzepte der Hamiltonschen Mechanik nutzt, um effizient aus komplexen Wahrscheinlichkeitsverteilungen zu sampeln. HMC erlaubt größere Sprünge im Parameterraum und reduziert die Autokorrelation zwischen aufeinanderfolgenden Samples, was die Konvergenz beschleunigt.}
}

\newglossaryentry{kuenstlicheintelligenz}{
  name={KI},
  description={\textit{Künstliche Intelligenz} ist Teilgebiet der Informatik, das sich mit der Entwicklung von Systemen beschäftigt, die menschenähnliche Fähigkeiten wie Lernen, Problemlösen oder Sprachverstehen zeigen}
}

\newglossaryentry{MAE}{
    name=MAE,
    description={\textit{Mean Absolute Error} mittlere absolute Abweichung zwischen Vorhersage und tatsächlichem Wert}
}

\newglossaryentry{machinelearning}{
  name={ML},
  description={\textit{Machine Learning} ist Teilgebiet der Künstlichen Intelligenz, das Systeme befähigt, Muster in Daten zu erkennen und daraus zu lernen, ohne explizit programmiert zu sein}
}

\newglossaryentry{MonteCarloSampling}{
  name={Monte-Carlo-Sampling},
  description={\textit{Monte-Carlo-Sampling} ist ein numerisches Verfahren, bei dem statistische Stichproben gezogen werden, um komplexe Integrale oder Wahrscheinlichkeitsverteilungen zu approximieren. Häufig verwendet in der Unsicherheitsquantifizierung und bei probabilistischen Modellen im Machine Learning.}
}

\newglossaryentry{ModelPredictiveControl}{
  name={MPC},
  description={\textit{Model Predictive Control} ist ein Regelungsansatz, der ein Modell des Systems verwendet, um zukünftige Systemzustände vorherzusagen und eine optimale Stellgrößenfolge zu berechnen. Dabei werden Einschränkungen berücksichtigt. MPC wird häufig in der Prozess- und Fahrzeugtechnik eingesetzt, insbesondere bei Systemen mit mehreren Eingangs- und Ausgangsgrößen.}
}

\newglossaryentry{neuronalesnetzwerk}{
  name={NN},
  description={\textit{Neuronales Netzwerk} ist ein Modell des maschinellen Lernens, das aus vielen miteinander verbundenen künstlichen Neuronen besteht. Inspiriert vom Aufbau biologischer Nervensysteme, kann ein neuronales Netzwerk komplexe Muster und Zusammenhänge in Daten erkennen und lernen. Es wird häufig in Bereichen wie Bild- und Spracherkennung, Prognosen oder Klassifikationen eingesetzt}
}

\newglossaryentry{NLL}{
    name=NLL,
    description={\textit{Negative Log-Likelihood} negative logarithmierte Likelihood der Daten unter dem Modell, oft verwendet als Loss-Funktion für probabilistische Modelle}
}

\newglossaryentry{Out-of-Distribution}{
  name={OOD},
  description={\textit{Out-Of-Distribution-Data} bezeichnet Datenpunkte, die aus einer anderen Verteilung stammen als derjenigen, auf der das Modell trainiert wurde, und deren Erkennung für die zuverlässige Unsicherheitsquantifizierung essenziell ist}
}

\newglossaryentry{relativeuncertaintyindex}{
  name={RUI},
  description={\textit{Relative Uncertainty Index} ist ein Maß für die Unsicherheitsbewertung im Verhältnis zur mittleren Vorhersage oder zu Referenzwerten}
}

\newglossaryentry{RMSE}{
    name=RMSE,
    description={\textit{Root Mean Squared Error} Maß für die durchschnittliche quadratische Abweichung zwischen Vorhersage und tatsächlichem Wert.}\)}
}

\newglossaryentry{surrogat}{
  name={Surrogatmodell},
  description={Ein Ersatzmodell zur Approximation komplexer Systeme}
}

\newglossaryentry{svi}{
  name={SVI},
  description={\textit{Stochastic Variational Inference} ist eine skalierbare Methode zur Approximation komplexer Posterior-Verteilungen. SVI nutzt stochastische Gradientenverfahren, um Variationsparameter auf großen Datensätzen effizient zu optimieren, oft eingesetzt in Bayesianischen Modellen.}
}

\newglossaryentry{variationalinference}{
  name={VI},
  description={\textit{Variational Inference} ist eine Approximation komplexer Wahrscheinlichkeitsverteilungen durch Optimierung einfacherer Verteilungen (Minimierung Kullback-Leibler)}
}
