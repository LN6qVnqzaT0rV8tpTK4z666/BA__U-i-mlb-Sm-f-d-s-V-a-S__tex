\PassOptionsToPackage{table,svgnames,dvipsnames}{xcolor}

\usepackage[utf8]{inputenc}
\usepackage[T1]{fontenc}
\usepackage[sc]{mathpazo}
\usepackage[ngerman]{babel}
\usepackage{booktabs}
\usepackage[autostyle]{csquotes}
\usepackage[%
  backend=biber,
  url=false,
  style=alphabetic,
  maxnames=4,
  minnames=3,
  maxbibnames=99,
  giveninits,
  uniquename=init,
  keywords=true]{biblatex} % TODO: adapt citation style
\usepackage[acronym]{glossaries} % Glossar
\usepackage{graphicx}
\usepackage{scrhack} % necessary for listings package
\usepackage{listings}
\usepackage{lstautogobble}
\usepackage{tikz}
\usepackage{pgfplots}
\usepackage{pgfplotstable}
\usepackage{booktabs}
\usepackage[final]{microtype}
\usepackage{caption}
\usepackage[hidelinks]{hyperref}
\usepackage{comment}
\usepackage{longtable} % Wenn Tabelle zu lang für Seite ist
\usepackage{array}     % Tabellensteuerung
\usepackage{enumitem}
\usepackage{silence}
\WarningFilter{latex}{Command \showhyphens has changed}
\WarningFilter{datatool-base}{No `datatool' support for dialect `ngerman'}

% \usepackage{parskip}
% \makeatletter
% \g@addto@macro\@makechapterhead{\@afterindentfalse}
% \makeatother

\makeglossaries

\bibliography{bibliography}
\addbibresource{bibliography.bib}

\setkomafont{disposition}{\normalfont\bfseries} % use serif font for headings
\linespread{1.05} % adjust line spread for mathpazo font

% Add table of contents to PDF bookmarks
\BeforeTOCHead[toc]{{\cleardoublepage\pdfbookmark[0]{\contentsname}{toc}}}

% Define TUM corporate design colors
% Taken from http://portal.mytum.de/corporatedesign/index_print/vorlagen/index_farben
\definecolor{TUMBlue}{HTML}{0065BD}
\definecolor{TUMSecondaryBlue}{HTML}{005293}
\definecolor{TUMSecondaryBlue2}{HTML}{003359}
\definecolor{TUMBlack}{HTML}{000000}
\definecolor{TUMWhite}{HTML}{FFFFFF}
\definecolor{TUMDarkGray}{HTML}{333333}
\definecolor{TUMGray}{HTML}{808080}
\definecolor{TUMLightGray}{HTML}{CCCCC6}
\definecolor{TUMAccentGray}{HTML}{DAD7CB}
\definecolor{TUMAccentOrange}{HTML}{E37222}
\definecolor{TUMAccentGreen}{HTML}{A2AD00}
\definecolor{TUMAccentLightBlue}{HTML}{98C6EA}
\definecolor{TUMAccentBlue}{HTML}{64A0C8}

% Settings for pgfplots
\pgfplotsset{compat=newest}
\pgfplotsset{
  % For available color names, see http://www.latextemplates.com/svgnames-colors
  cycle list={TUMBlue\\TUMAccentOrange\\TUMAccentGreen\\TUMSecondaryBlue2\\TUMDarkGray\\},
}

% Settings for lstlistings
\lstset{%
  basicstyle=\ttfamily,
  columns=fullflexible,
  autogobble,
  keywordstyle=\bfseries\color{TUMBlue},
  stringstyle=\color{TUMAccentGreen}
}

\addto\captionsngerman{%
  \renewcommand{\contentsname}{Inhaltsverzeichnis}
  \renewcommand{\listfigurename}{Liste der Abbildungen}
  \renewcommand{\listtablename}{Liste der Tabellen}
  \renewcommand{\abstractname}{Kurzfassung}
  \renewcommand{\bibname}{Literaturverzeichnis}
  \renewcommand{\glossaryname}{Glossar}
  \renewcommand{\acronymname}{Abkürzungsverzeichnis}
  \renewcommand{\indexname}{Stichwortverzeichnis}
}

\newglossaryentry{Aleatorische Unsicherheit}{
  name=AC,
  description={Datenunsicherheit}
}

\newglossaryentry{Autonomous Underwater Vehicle}{
  name=AUV,
  description={Autonomes Unterwasser Fahrzeug}
}

\newglossaryentry{Bundesministerium für Bildung und Forschung}{
  name=BMBF,
  description={Bundesministerium für Bildung und Forschung}
}

\newglossaryentry{Bayesianische neuronale Netze}{
  name=BNN,
  description={Bayesianische neuronale Netze, die Unsicherheit in den Modellparametern durch probabilistische Inferenz berücksichtigen, um eine Verteilung der möglichen Modellparameter statt eines festen Satzes von Parametern zu lernen}
}

\newglossaryentry{Conformal Prediction}{
  name=CP,
  description={Ein Verfahren zur Unsicherheitsquantifizierung, das es ermöglicht, Vorhersagen mit einer kontrollierten Fehlerwahrscheinlichkeit zu treffen. Conformal Prediction erzeugt für jede Vorhersage ein Konfidenzintervall, das die Unsicherheit in den Modellvorhersagen darstellt. Es basiert auf der Idee, dass die Vorhersage als "konform" zu den Trainingsdaten betrachtet wird, wenn sie eine bestimmte Konsistenzbedingung erfüllt. Dieses Verfahren ist besonders nützlich in Situationen, in denen die Datenverteilung unbekannt oder nicht unabhängig identisch verteilt ist}
}

\newglossaryentry{Deutschen Zentrum für Luft- und Raumfahrt}{
  name=DLR,
  description={Deutsches Zentrum für Luft- und Raumfahrt}
}

\newglossaryentry{Deutschen Allianz für Meeresforschung}{
  name=DAM,
  description={Deutschen Allianz für Meeresforschung}
}

\newglossaryentry{Epistemische Unsicherheit}{
  name=EC,
  description={Modellunsicherheit}
}

\newglossaryentry{Evidenzbasierte neuronale Netze}{
  name=ENN,
  description={Evidenzbasierte neuronale Netze, die Unsicherheit nicht nur in den Modellparametern, sondern auch in den Daten selbst durch die Verwendung einer evidenzbasierten Inferenztechnik berücksichtigen. Statt eine feste Wahrscheinlichkeitsverteilung der Modellparameter zu lernen, modellieren ENNs die Unsicherheit als Evidenz, die die Unsicherheit sowohl der Daten als auch der Modellparameter direkt widerspiegelt. Diese Methode ermöglicht eine robustere Quantifizierung von Unsicherheit und verbessert die Handhabung von Unsicherheit in realen Anwendungsszenarien}
}

\newglossaryentry{Gaußsche Prozessregression}{
  name=GPR,
  description={Nichtparametrisches, probabilistisches Modell zur Vorhersage,  Quantifizierung von Unsicherheiten in Regressionen (Vorhersagen als Verteilungen statt als Punktwerte)}
}

\newglossaryentry{surrogat}{
  name=Surrogatmodell,
  description={Ein Ersatzmodell zur Approximation komplexer Systeme}
}